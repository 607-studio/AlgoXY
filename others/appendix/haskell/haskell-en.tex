\ifx\wholebook\relax \else
% ------------------------

\documentclass{article}
%------------------- Other types of document example ------------------------
%
%\documentclass[twocolumn]{IEEEtran-new}
%\documentclass[12pt,twoside,draft]{IEEEtran}
%\documentstyle[9pt,twocolumn,technote,twoside]{IEEEtran}
%
%-----------------------------------------------------------------------------
%\input{../../../common.tex}
%
% loading packages
%

\RequirePackage{ifpdf}
\RequirePackage{ifxetex}

%
%
\ifpdf
  \RequirePackage[pdftex,%
       bookmarksnumbered,%
              colorlinks,%
          linkcolor=blue,%
              hyperindex,%
        plainpages=false,%
       pdfstartview=FitH]{hyperref}
\else\ifxetex
  \RequirePackage[bookmarksnumbered,%
               colorlinks,%
           linkcolor=blue,%
               hyperindex,%
         plainpages=false,%
        pdfstartview=FitH]{hyperref}
\else
  \RequirePackage[dvipdfm,%
        bookmarksnumbered,%
               colorlinks,%
           linkcolor=blue,%
               hyperindex,%
         plainpages=false,%
        pdfstartview=FitH]{hyperref}
\fi\fi
%\usepackage{hyperref}

% other packages
%--------------------------------------------------------------------------
\usepackage{graphicx, color}
\usepackage{subfig}
\usepackage{tikz}
\usetikzlibrary{matrix,positioning}

\usepackage{amsmath, amsthm, amssymb} % for math
\usepackage{exercise} % for exercise
\usepackage{import} % for nested input

%
% for programming
%
\usepackage{verbatim}
\usepackage{listings}
\usepackage{lipsum}
%\usepackage{algorithmic} %old version; we can use algorithmicx instead
\usepackage{algorithm}
\usepackage[noend]{algpseudocode} %for pseudo code, include algorithmicsx automatically
\usepackage{appendix}
\usepackage{makeidx} % for index support
\usepackage{titlesec}

\usepackage{fontspec}
\usepackage{xunicode}
\usepackage{fontenc}
\usepackage{textcomp}
\usepackage{url}
\usepackage{courier}

\titleformat{\paragraph}
{\normalfont\normalsize\bfseries}{\theparagraph}{1em}{}
\titlespacing*{\paragraph}
{0pt}{3.25ex plus 1ex minus .2ex}{1.5ex plus .2ex}

\lstdefinelanguage{Smalltalk}{
  morekeywords={self,super,true,false,nil,thisContext}, % This is overkill
  morestring=[d]',
  morecomment=[s]{"}{"},
  alsoletter={\#:},
  escapechar={!},
  literate=
    {BANG}{!}1
    {UNDERSCORE}{\_}1
    {\\st}{Smalltalk}9 % convenience -- in case \st occurs in code
    % {'}{{\textquotesingle}}1 % replaced by upquote=true in \lstset
    {_}{{$\leftarrow$}}1
    {>>>}{{\sep}}1
    {^}{{$\uparrow$}}1
    {~}{{$\sim$}}1
    {-}{{\sf -\hspace{-0.13em}-}}1  % the goal is to make - the same width as +
    %{+}{\raisebox{0.08ex}{+}}1		% and to raise + off the baseline to match -
    {-->}{{\quad$\longrightarrow$\quad}}3
	, % Don't forget the comma at the end!
  tabsize=2
}[keywords,comments,strings]

%% \lstdefinestyle{Haskell}{
%%   flexiblecolumns=false,
%%   basewidth={0.5em,0.45em},
%%   morecomment=[l]--,
%%   literate={+}{{$+$}}1 {/}{{$/$}}1 {*}{{$*$}}1 {=}{{$=$}}1
%%            {>}{{$>$}}1 {<}{{$<$}}1 {\\}{{$\lambda$}}1
%%            {\\\\}{{\char`\\\char`\\}}1
%%            {->}{{$\rightarrow$}}2 {>=}{{$\geq$}}2 {<-}{{$\leftarrow$}}2
%%            {<=}{{$\leq$}}2 {=>}{{$\Rightarrow$}}2
%%            {\ .}{{$\circ$}}2 {\ .\ }{{$\circ$}}2
%%            {>>}{{>>}}2 {>>=}{{>>=}}2
%%            {|}{{$\mid$}}1
%% }

% For better Haskell code outlook
\lstdefinelanguage{Haskell}{
  flexiblecolumns=false,
  basewidth={0.5em,0.45em},
  morecomment=[l]--,
  morekeywords={case, class, do, else, True, False, if, import,
    instance, module, type, data, deriving, where},
  literate={+}{{$+$}}1 {/}{{$/$}}1 {*}{{$*$}}1 {=}{{$=$}}1
           {>}{{$>$}}1 {<}{{$<$}}1 {\\}{{$\lambda$}}1
           {\\\\}{{\char`\\\char`\\}}1
           {->}{{$\rightarrow$}}2 {>=}{{$\geq$}}2 {<-}{{$\leftarrow$}}2
           {<=}{{$\leq$}}2 {=>}{{$\Rightarrow$}}2
           {\ .}{{$\circ$}}2 {\ .\ }{{$\circ$}}2
           {>>}{{>>}}2 {>>=}{{>>=}}2
           {|}{{$\mid$}}1
}[keywords,comments,strings]

% "define" Scala
\lstdefinelanguage{Scala}{
  morekeywords={abstract,case,catch,class,def,%
    do,else,extends,false,final,finally,%
    for,if,implicit,import,match,mixin,%
    new,null,object,override,package,%
    private,protected,requires,return,sealed,%
    super,this,throw,trait,true,try,%
    type,val,var,while,with,yield},
  otherkeywords={=>,<-,<\%,<:,>:,\#,@},
  sensitive=true,
  morecomment=[l]{//},
  morecomment=[n]{/*}{*/},
  morestring=[b]",
  morestring=[b]',
  morestring=[b]"""
}

\lstloadlanguages{Java, C, C++, Lisp, Haskell, Python, Smalltalk, Scala}

\lstset{
  basicstyle=\small,
  commentstyle=\rmfamily,
  %keywordstyle=\bfseries,
  texcl=true,
  showstringspaces = false,
  upquote=true,
  flexiblecolumns=false
}

\renewcommand{\lstlistingname}{Code}

% ======================================================================

\def\BibTeX{{\rm B\kern-.05em{\sc i\kern-.025em b}\kern-.08em
    T\kern-.1667em\lower.7ex\hbox{E}\kern-.125emX}}

%
% mathematics
%
\newcommand{\be}{\begin{equation}}
\newcommand{\ee}{\end{equation}}
\newcommand{\bmat}[1]{\left( \begin{array}{#1} }
\newcommand{\emat}{\end{array} \right) }
\newcommand{\VEC}[1]{\mbox{\boldmath $#1$}}

% numbered equation array
\newcommand{\bea}{\begin{eqnarray}}
\newcommand{\eea}{\end{eqnarray}}

% equation array not numbered
\newcommand{\bean}{\begin{eqnarray*}}
\newcommand{\eean}{\end{eqnarray*}}

\newtheorem{theorem}{Theorem}[section]
\newtheorem{lemma}[theorem]{Lemma}
\newtheorem{proposition}[theorem]{Proposition}
\newtheorem{corollary}[theorem]{Corollary}

\setcounter{tocdepth}{4}
\setcounter{secnumdepth}{4}


\setcounter{page}{1}

\begin{document}

%--------------------------

% ================================================================
%                 COVER PAGE
% ================================================================

\title{Haskell examples}

\author{Larry~LIU~Xinyu
\thanks{{\bfseries Larry LIU Xinyu } \newline
  Email: liuxinyu95@gmail.com \newline}
  }

\maketitle
\fi

\markboth{Haskell Examples}{Elementary Algorithms}

\ifx\wholebook\relax
\chapter{Haskell examples}
\numberwithin{Exercise}{chapter}
\fi

This appendix list the example Haskell code for reference.

\section{Preface}

\subsection{Minimum free number puzzle}

\lstset{language=Haskell, frame=single}
\begin{lstlisting}[caption=The divide and conquer solution to the mininum free number puzzle.]
import Data.List

minFree xs = bsearch xs 0 (length xs - 1)

bsearch xs l u | xs == [] = l
               | length as == m - l + 1 = bsearch bs (m + 1) u
               | otherwise = bsearch as l m
    where
      m = (l + u) `div` 2
      (as, bs) = partition (<=m) xs
\end{lstlisting}

\subsection{Regular number (Hamming number)}
Find the k-th number that only contains factor 2, 3, or 5. Below Haskell
example based on the idea of infinite stream and lazy evaluation.

\begin{lstlisting}[caption=Recursive Hamming number definition]
merge (x:xs) (y:ys) | x <y = x : merge xs (y:ys)
                    | x == y = x : merge xs ys
                    | otherwise = y : merge (x:xs) ys

ns = 1:merge (map (*2) ns) (merge (map (*3) ns) (map (*5) ns))
\end{lstlisting}

Below line gives the 1500th Regular number.
\begin{verbatim}
last $ take 1500 ns
\end{verbatim}

Alternatively, we can use three queues to populate the regular numbers.

\begin{lstlisting}[caption=Populate Hamming number with queues]
ks 1 xs _ = xs
ks n xs (q2, q3, q5) = ks (n-1) (xs++[x]) update
    where
      x = minimum $ map head [q2, q3, q5]
      update | x == head q2 = ((tail q2)++[x*2], q3++[x*3], q5++[x*5])
             | x == head q3 = (q2, (tail q3)++[x*3], q5++[x*5])
             | otherwise = (q2, q3, (tail q5)++[x*5])

takeN n = ks n [1] ([2], [3], [5])
\end{lstlisting}

Run \texttt{last \$ takeN 1500} will generate the 1500th number 859963392.

\section{Binary Search Tree}

\begin{lstlisting}[caption=Algebraic data type definition of BST]
data Tree a = Empty
            | Node (Tree a) a (Tree a)
\end{lstlisting}

\begin{lstlisting}[caption=BST insertion]
insert Empty k = Node Empty k Empty
insert (Node l x r) k | k < x = Node (insert l k) x r
                      | otherwise = Node l x (insert r k)
\end{lstlisting}

\begin{lstlisting}[caption=Flattern BST to ordered list.]
toList Empty = []
toList (Node l x r) = toList l ++ [x] ++ toList r
\end{lstlisting}

\begin{lstlisting}[caption=look up]
lookup Empty _ = Empty
lookup t@(Node l k r) x | k == x = t
                        | x < k = lookup l x
                        | otherwise = lookup r x
\end{lstlisting}

\begin{lstlisting}[caption=BST deletion]
delete Empty _ = Empty
delete (Node l k r) x | x < k = (Node (delete l x) k r)
                      | x > k = (Node l k (delete r x))
                      -- x == k
                      | isEmpty l = r
                      | isEmpty r = l
                      | otherwise = (Node l k' (delete r k'))
                          where k' = min r
\end{lstlisting}

\section{Insertion Sort}

\begin{lstlisting}[caption=ordered insertion]
insert [] x = [x]
insert (y:ys) x = if x < y then x:y:ys else y:insert ys x
\end{lstlisting}

\subsection{Insertion Sort}

\begin{lstlisting}[caption=recursive insertion sort]
isort [] = []
isort (x:xs) = insert (isort xs) x
\end{lstlisting}

\begin{lstlisting}[caption=define insertion sort with fold]
isort = foldl insert []
\end{lstlisting}

\section{Red-black tree}

\begin{lstlisting}[caption=Red-black tree insertion with pattern matching]
data Color = R | B
data RBTree a = Empty
              | Node Color (RBTree a) a (RBTree a)

insert::(Ord a)=>RBTree a -> a -> RBTree a
insert t x = makeBlack $ ins t where
    ins Empty = Node R Empty x Empty
    ins (Node color l k r)
        | x < k     = balance color (ins l) k r
        | otherwise = balance color l k (ins r)
    makeBlack(Node _ l k r) = Node B l k r

balance::Color -> RBTree a -> a -> RBTree a -> RBTree a
balance B (Node R (Node R a x b) y c) z d =
        Node R (Node B a x b) y (Node B c z d)
balance B (Node R a x (Node R b y c)) z d =
        Node R (Node B a x b) y (Node B c z d)
balance B a x (Node R b y (Node R c z d)) =
        Node R (Node B a x b) y (Node B c z d)
balance B a x (Node R (Node R b y c) z d) =
        Node R (Node B a x b) y (Node B c z d)
balance color l k r = Node color l k r

\end{lstlisting}

\subsection{deletion}



\begin{lstlisting}[caption=Red-black tree deletion algorithm with the concept of 'doubly black'.]
data Color = R | B | BB deriving (Show, Eq) -- BB is doubly black, used for deletion
data RBTree a = Empty
              | Node Color (RBTree a) a (RBTree a)
              | BBEmpty -- doubly black empty

min::RBTree a -> a
min (Node _ Empty x _) = x
min (Node _ l _ _) = min l

isEmpty :: (RBTree a) -> Bool
isEmpty Empty = True
isEmpty _ = False

delete::(Ord a)=>RBTree a -> a -> RBTree a
delete t x = blackenRoot (del t x) where
    del Empty _ = Empty
    del (Node color l k r) x
        | x < k = fixDB color (del l x) k r
        | x > k = fixDB color l k (del r x)
        -- x == k, delete this node
        | isEmpty l = if color==B then makeBlack r else r
        | isEmpty r = if color==B then makeBlack l else l
        | otherwise = fixDB color l k' (del r k') where k'= min r
    blackenRoot (Node _ l k r) = Node B l k r
    blackenRoot _ = Empty

makeBlack::RBTree a -> RBTree a
makeBlack (Node B l k r) = Node BB l k r -- doubly black
makeBlack (Node _ l k r) = Node B l k r
makeBlack Empty = BBEmpty
makeBlack t = t

fixDB::Color -> RBTree a -> a -> RBTree a -> RBTree a
-- the sibling is black, and it has one red child
fixDB color a@(Node BB _ _ _) x (Node B (Node R b y c) z d) =
      Node color (Node B (makeBlack a) x b) y (Node B c z d)
fixDB color BBEmpty x (Node B (Node R b y c) z d) =
      Node color (Node B Empty x b) y (Node B c z d)
fixDB color a@(Node BB _ _ _) x (Node B b y (Node R c z d)) =
      Node color (Node B (makeBlack a) x b) y (Node B c z d)
fixDB color BBEmpty x (Node B b y (Node R c z d)) =
      Node color (Node B Empty x b) y (Node B c z d)
fixDB color (Node B a x (Node R b y c)) z d@(Node BB _ _ _) =
      Node color (Node B a x b) y (Node B c z (makeBlack d))
fixDB color (Node B a x (Node R b y c)) z BBEmpty =
      Node color (Node B a x b) y (Node B c z Empty)
fixDB color (Node B (Node R a x b) y c) z d@(Node BB _ _ _) =
      Node color (Node B a x b) y (Node B c z (makeBlack d))
fixDB color (Node B (Node R a x b) y c) z BBEmpty =
      Node color (Node B a x b) y (Node B c z Empty)
-- the sibling is red
fixDB B a@(Node BB _ _ _) x (Node R b y c) = fixDB B (fixDB R a x b) y c
fixDB B a@BBEmpty x (Node R b y c) = fixDB B (fixDB R a x b) y c
fixDB B (Node R a x b) y c@(Node BB _ _ _) = fixDB B a x (fixDB R b y c)
fixDB B (Node R a x b) y c@BBEmpty = fixDB B a x (fixDB R b y c)
-- the sibling and its 2 children are all black, propagate the blackness up
fixDB color a@(Node BB _ _ _) x (Node B b y c) =
      makeBlack (Node color (makeBlack a) x (Node R b y c))
fixDB color BBEmpty x (Node B b y c) =
      makeBlack (Node color Empty x (Node R b y c))
fixDB color (Node B a x b) y c@(Node BB _ _ _) = m
      akeBlack (Node color (Node R a x b) y (makeBlack c))
fixDB color (Node B a x b) y BBEmpty =
      makeBlack (Node color (Node R a x b) y Empty)
-- otherwise
fixDB color l k r = Node color l k r
\end{lstlisting}

\section{AVL tree}

AVL tree is defined as Algebraic data type. It's either empty or contains four
components: the left and right sub trees, the key, and the balance factor.

\begin{lstlisting}[caption=Algebraic AVL tree definition]
import Prelude hiding (min)

data AVLTree a = Empty
               | Br (AVLTree a) a (AVLTree a) Int

isEmpty Empty = True
isEmpty _ = False

min :: AVLTree a -> a
min (Br Empty x _ _) = x
min (Br l _ _ _) = min l
\end{lstlisting}

\begin{lstlisting}[caption=AVL tree insertion algorithm with Pattern matching approach.]
insert::(Ord a)=>AVLTree a -> a -> AVLTree a
insert t x = fst $ ins t where
    -- result of ins is a pair (t, d), t: tree, d: increment of height
    ins Empty = (Br Empty x Empty 0, 1)
    ins (Br l k r d)
        | x < k     = node (ins l) k (r, 0) d
        | x == k    = (Br l k r d, 0)
        | otherwise = node (l, 0) k (ins r) d

-- params: (left, increment on left) key (right, increment on right)
node::(AVLTree a, Int) -> a -> (AVLTree a, Int) -> Int -> (AVLTree a, Int)
node (l, dl) k (r, dr) d = balance (Br l k r d', delta) where
    d' = d + dr - dl
    delta = deltaH d d' dl dr

-- delta(Height) = max(|R'|, |L'|) - max (|R|, |L|)
deltaH :: Int -> Int -> Int -> Int -> Int
deltaH d d' dl dr
       | d >=0 && d' >=0 = dr
       | d <=0 && d' >=0 = d+dr
       | d >=0 && d' <=0 = dl - d
       | otherwise = dl

balance :: (AVLTree a, Int) -> (AVLTree a, Int)
balance (Br (Br (Br a x b dx) y c (-1)) z d (-2), dH) =
    (Br (Br a x b dx) y (Br c z d 0) 0, dH-1)
balance (Br a x (Br b y (Br c z d dz)    1)    2, dH) =
    (Br (Br a x b 0) y (Br c z d dz) 0, dH-1)
balance (Br (Br a x (Br b y c dy)    1) z d (-2), dH) =
    (Br (Br a x b dx') y (Br c z d dz') 0, dH-1) where
        dx' = if dy ==  1 then -1 else 0
        dz' = if dy == -1 then  1 else 0
balance (Br a x (Br (Br b y c dy) z d (-1))    2, dH) =
    (Br (Br a x b dx') y (Br c z d dz') 0, dH-1) where
        dx' = if dy ==  1 then -1 else 0
        dz' = if dy == -1 then  1 else 0
balance (t, d) = (t, d)
\end{lstlisting}

If we extend the height increment to nagative value, most functions we developed
in insertion are still applicable. We need add two special patterns when fixing
the balance.

\begin{lstlisting}[caption=AVL tree deletion]
delete::(Ord a) => AVLTree a -> a -> AVLTree a
delete t x = fst $ del t x where
  -- result is a pair (t, d), t: tree, d: decrement in height
  del Empty _ = (Empty, 0)
  del (Br l k r d) x
    | x < k = node (del l x) k (r, 0) d
    | x > k = node (l, 0) k (del r x) d
    -- x == k, delete this node
    | isEmpty l = (r, -1)
    | isEmpty r = (l, -1)
    | otherwise = node (l, 0) k' (del r k') d where k' = min r

balance :: (AVLTree a, Int) -> (AVLTree a, Int)
balance (Br (Br (Br a x b dx) y c (-1)) z d (-2), dH) =
    (Br (Br a x b dx) y (Br c z d 0) 0, dH-1)
balance (Br a x (Br b y (Br c z d dz)    1)    2, dH) =
    (Br (Br a x b 0) y (Br c z d dz) 0, dH-1)
balance (Br (Br a x (Br b y c dy)    1) z d (-2), dH) =
    (Br (Br a x b dx') y (Br c z d dz') 0, dH-1) where
        dx' = if dy ==  1 then -1 else 0
        dz' = if dy == -1 then  1 else 0
balance (Br a x (Br (Br b y c dy) z d (-1))    2, dH) =
    (Br (Br a x b dx') y (Br c z d dz') 0, dH-1) where
        dx' = if dy ==  1 then -1 else 0
        dz' = if dy == -1 then  1 else 0
-- Delete specific fixing
balance (Br (Br a x b dx) y c (-2), dH) =
    (Br a x (Br b y c (-1)) (dx+1), dH)
balance (Br a x (Br b y c dy)    2, dH) =
    (Br (Br a x b    1) y c (dy-1), dH)
balance (t, d) = (t, d)
\end{lstlisting}

\section{Radix Tree}

\subsection{Integer Trie}
Little endian integer trie.

\lstset{language=Haskell}
\begin{lstlisting}[caption=Algebraic definition of integer trie]
data IntTrie a = Empty
               | Branch (IntTrie a) (Maybe a) (IntTrie a)
\end{lstlisting}

\lstset{language=Haskell}
\begin{lstlisting}[caption=Integer trie insertion]
insert t 0 x = Branch (left t) (Just x) (right t)
insert t k x
    | even k = Branch (insert (left t) (k `div` 2) x) (value t) (right t)
    | otherwise = Branch (left t) (value t) (insert (right t) (k `div` 2) x)

left (Branch l _ _) = l
left Empty = Empty

right (Branch _ _ r) = r
right Empty = Empty

value (Branch _ v _) = v
value Empty = Nothing
\end{lstlisting}

\lstset{language=Haskell}
\begin{lstlisting}[caption=Integer trie look up]
search Empty k = Nothing
search t 0 = value t
search t k = if even k then search (left t) (k `div` 2)
             else search (right t) (k `div` 2)
\end{lstlisting}

\subsection{Integer prefix tree}

\lstset{language=Haskell}
\begin{lstlisting}[caption=Integer prefix tree definition]
type Key = Int
type Prefix = Int
type Mask = Int

data IntTree a = Empty
               | Leaf Key a
               | Branch Prefix Mask (IntTree a) (IntTree a)
\end{lstlisting}

\lstset{language=Haskell}
\begin{lstlisting}[caption=Integer prefix tree insertion.]
import Data.Bits

insert t k x
   = case t of
       Empty -> Leaf k x
       Leaf k' x' -> if k==k' then Leaf k x
                     else join k (Leaf k x) k' t -- t@(Leaf k' x')
       Branch p m l r
          | match k p m -> if zero k m
                           then Branch p m (insert l k x) r
                           else Branch p m l (insert r k x)
          | otherwise -> join k (Leaf k x) p t -- t@(Branch p m l r)

join p1 t1 p2 t2 = if zero p1 m then Branch p m t1 t2
                                else Branch p m t2 t1
    where
      (p, m) = lcp p1 p2

lcp :: Prefix -> Prefix -> (Prefix, Mask)
lcp p1 p2 = (p, m) where
    m = bit (highestBit (p1 `xor` p2))
    p = mask p1 m

highestBit x = if x == 0 then 0 else 1 + highestBit (shiftR x 1)

mask x m = (x .&. complement (m-1)) -- complement means bit-wise not.

zero x m = x .&. (shiftR m 1) == 0

match k p m = (mask k m) == p
\end{lstlisting}

\ifx\wholebook\relax \else
\end{document}
\fi
