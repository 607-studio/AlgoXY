\ifx\wholebook\relax \else
% ------------------------

\documentclass[UTF8]{article}
%------------------- Other types of document example ------------------------
%
%\documentclass[twocolumn]{IEEEtran-new}
%\documentclass[12pt,twoside,draft]{IEEEtran}
%\documentstyle[9pt,twocolumn,technote,twoside]{IEEEtran}
%
%-----------------------------------------------------------------------------
%
% loading packages
%

\RequirePackage{ifpdf}
\RequirePackage{ifxetex}

%
%
\ifpdf
  \RequirePackage[pdftex,%
       bookmarksnumbered,%
              colorlinks,%
          linkcolor=blue,%
              hyperindex,%
        plainpages=false,%
       pdfstartview=FitH]{hyperref}
\else\ifxetex
  \RequirePackage[bookmarksnumbered,%
               colorlinks,%
           linkcolor=blue,%
               hyperindex,%
         plainpages=false,%
        pdfstartview=FitH]{hyperref}
\else
  \RequirePackage[dvipdfm,%
        bookmarksnumbered,%
               colorlinks,%
           linkcolor=blue,%
               hyperindex,%
         plainpages=false,%
        pdfstartview=FitH]{hyperref}
\fi\fi
%\usepackage{hyperref}

% other packages
%--------------------------------------------------------------------------
\usepackage{graphicx, color}
\usepackage{subfig}
\usepackage{multicol}
\usepackage{tikz}
\usetikzlibrary{matrix,positioning,shapes}

\usepackage{amsmath, amsthm, amssymb} % for math
\usepackage{exercise} % for exercise
\usepackage{import} % for nested input

%
% for programming
%
\usepackage{verbatim}
\usepackage{fancyvrb}
\usepackage{listings}
%\usepackage{algorithmic} %old version; we can use algorithmicx instead
%\usepackage[plain]{algorithm} %remove rule (horizontal line on top/below the algorithm
\usepackage{algorithm} %to remove rules change to \usepackage[plain]{algorithm}
%\usepackage{algorithm2e}
\usepackage[noend]{algpseudocode} %for pseudo code, include algorithmicsx automatically
\usepackage{appendix}
\usepackage{makeidx} % for index support
\usepackage{titlesec}

\usepackage[cm-default]{fontspec}
\usepackage{xunicode}
%\usepackage{fontenc}
\usepackage{textcomp}
\usepackage{url}

% detect and select Chinese font
% ------------------------------
% the following cmd can list all availabe Chinese fonts in host.
% fc-list :lang=zh
\def\myfont{STSong}  % Under Mac OS X
\def\linuxfallback{WenQuanYi Micro Hei} % Under Linux
\def\winfallback{SimSun} % Under Windows
\suppressfontnotfounderror1 % Avoid setting exit code (error level) to break make process
\count255=\interactionmode
\batchmode
\font\foo="\myfont"\space at 10pt
\ifx\foo\nullfont
  \font\foo = "\linuxfallback"\space at 10pt
  \ifx\foo\nullfont
    \font\foo = "\winfallback"\space at 10pt
    \ifx\foo\nullfont
      \errorstopmode
      \errmessage{no suitable Chinese font found}
    \else
      \let\myfont=\winfallback % Windows
    \fi
  \else
    \let\myfont=\linuxfallback % Linux
  \fi
\fi
\interactionmode=\count255
\setmainfont[Mapping=tex-text]{\myfont}
\setmonofont[Scale=MatchLowercase]{Monaco}   % 英文等宽字体

\XeTeXlinebreaklocale "zh"  % to solve the line breaking issue
\XeTeXlinebreakskip = 0pt plus 1pt minus 0.1pt

\titleformat{\paragraph}
{\normalfont\normalsize\bfseries}{\theparagraph}{1em}{}
\titlespacing*{\paragraph}
{0pt}{3.25ex plus 1ex minus .2ex}{1.5ex plus .2ex}

\lstdefinelanguage{Smalltalk}{
  morekeywords={self,super,true,false,nil,thisContext}, % This is overkill
  morestring=[d]',
  morecomment=[s]{"}{"},
  alsoletter={\#:},
  escapechar={!},
  literate=
    {BANG}{!}1
    {UNDERSCORE}{\_}1
    {\\st}{Smalltalk}9 % convenience -- in case \st occurs in code
    % {'}{{\textquotesingle}}1 % replaced by upquote=true in \lstset
    {_}{{$\leftarrow$}}1
    {>>>}{{\sep}}1
    {^}{{$\uparrow$}}1
    {~}{{$\sim$}}1
    {-}{{\sf -\hspace{-0.13em}-}}1  % the goal is to make - the same width as +
    %{+}{\raisebox{0.08ex}{+}}1		% and to raise + off the baseline to match -
    {-->}{{\quad$\longrightarrow$\quad}}3
	, % Don't forget the comma at the end!
  tabsize=2
}[keywords,comments,strings]

% for literate Haskell code
\lstdefinestyle{Haskell}{
  flexiblecolumns=false,
  basewidth={0.5em,0.45em},
  morecomment=[l]--,
  literate={+}{{$+$}}1 {/}{{$/$}}1 {*}{{$*$}}1 {=}{{$=$}}1
           {>}{{$>$}}1 {<}{{$<$}}1 {\\}{{$\lambda$}}1
           {\\\\}{{\char`\\\char`\\}}1
           {->}{{$\rightarrow$}}2 {>=}{{$\geq$}}2 {<-}{{$\leftarrow$}}2
           {<=}{{$\leq$}}2 {=>}{{$\Rightarrow$}}2
           {\ .}{{$\circ$}}2 {\ .\ }{{$\circ$}}2
           {>>}{{>>}}2 {>>=}{{>>=}}2
           {|}{{$\mid$}}1
}

\lstloadlanguages{C, C++, Lisp, Haskell, Python, Smalltalk}

\lstset{
  basicstyle=\small\ttfamily,
  commentstyle=\rmfamily,
  texcl=true,
  showstringspaces = false,
  upquote=true,
  flexiblecolumns=false
}

% ======================================================================

\def\BibTeX{{\rm B\kern-.05em{\sc i\kern-.025em b}\kern-.08em
    T\kern-.1667em\lower.7ex\hbox{E}\kern-.125emX}}

%
% mathematics
%
\newcommand{\be}{\begin{equation}}
\newcommand{\ee}{\end{equation}}
\newcommand{\bmat}[1]{\left( \begin{array}{#1} }
\newcommand{\emat}{\end{array} \right) }
\newcommand{\VEC}[1]{\mbox{\boldmath $#1$}}

% numbered equation array
\newcommand{\bea}{\begin{eqnarray}}
\newcommand{\eea}{\end{eqnarray}}

% equation array not numbered
\newcommand{\bean}{\begin{eqnarray*}}
\newcommand{\eean}{\end{eqnarray*}}

\newtheorem{theorem}{定理}[section]
\newtheorem{lemma}[theorem]{引理}
\newtheorem{proposition}[theorem]{Proposition}
\newtheorem{corollary}[theorem]{Corollary}

% 中文书籍设置
% ====================================
\renewcommand\contentsname{目\ 录}
%\renewcommand\listfigurename{插图目录}
%\renewcommand\listtablename{表格目录}
\renewcommand\figurename{图}
\renewcommand\tablename{表}
\renewcommand\proofname{证明}
\renewcommand\ExerciseName{练习}
%\renewcommand{\algorithmcfname}{算法}

\ifx\wholebook\relax
\renewcommand\bibname{参\ 考\ 文\ 献}                    %book类型
%\newtheorem{Definition}[Theorem]{定义}
\newtheorem{Theorem}{定理}[chapter]
\newtheorem{example}{例题}[chapter]
\else
\renewcommand\refname{参\ 考\ 文\ 献}
\fi

%\setcounter{secnumdepth}{4}
\titleformat{\chapter}
  {\normalfont\bfseries\Large}
  {第\arabic{chapter}章}
  {12pt}{\Large}
%% \titleformat{\subsection}
%%   {\normalfont\bfseries\large}
%%   {\CJKnumber{\arabic{subsection}}、}
%%   {12pt}{\large}
%% \titleformat{\subsubsection}
%%   {\normalfont\bfseries\normalsize}
%%   {\arabic{subsubsection}.}
%%   {12pt}{\normalsize}

%\renewcommand{\baselinestretch}{1.5}                        %文章行间距为1.5倍。

\makeatletter
\newcommand{\verbatimfont}[1]{\renewcommand{\verbatim@font}{\ttfamily#1}}
\makeatother

\setcounter{tocdepth}{4}
\setcounter{secnumdepth}{4}

%\verbatimfont{\footnotesize}


\setcounter{page}{1}

\begin{document}

%--------------------------

% ================================================================
%                 COVER PAGE
% ================================================================

\title{红黑树的命令式删除算法}

\author{刘新宇
\thanks{{\bfseries 刘新宇} \newline
  Email: liuxinyu95@gmail.com \newline}
  }

\maketitle
\fi

\markboth{红黑树}{初等算法}

\ifx\wholebook\relax
\chapter{红黑树的命令式删除算法}
\numberwithin{Exercise}{chapter}
\fi

% ================================================================
%                 Introduction
% ================================================================
\index{红黑树!命令式删除}

本附录包含红黑树的命令式删除算法。我们需要在普通二叉搜索树删除算法的基础上,通过旋转和重新染色恢复红黑树的性质,以保持树的平衡。我们在红黑树一章中指出,当删除黑色节点时,会破坏红黑树的第五条性质。使得某一路径上的黑色节点数目减少。为此,我们引入“双重黑色”节点,来保持所删除路径上的黑色节点数目不变。

\section{双重黑色}

为了支持“双重黑色”的节点,我们需要增加颜色的定义。如下面的C++例子代码所示。

\lstset{language=C++}
\begin{lstlisting}
enum class Color { RED, BLACK, DOUBLY_BLACK };
\end{lstlisting}

在删除一个节点时,我们复用二叉搜索树的删除算法,并记录被删除节点的父节点。如果被删除节点的颜色是黑色我们需要通过处理保持黑色的属性,然后再进行进一步修复。

\begin{algorithmic}[1]
\Function{Delete}{$T, x$}
  \State $p \gets$ \Call{Parent}{$x$}
  \State $q \gets$ NIL
  \If{\Call{Left}{$x$} = NIL}
    \State $q \gets$ \Call{Right}{$x$}
    \State replace $x$ with \Call{Right}{$x$}
  \ElsIf{\Call{Right}{$x$} = NIL}
    \State $q \gets$ \Call{Left}{$x$}
    \State replace $x$ with \Call{Left}{$x$}
  \Else
    \State $y \gets$ \textproc{Min}(\Call{Right}{$x$})
    \State $p \gets$ \Call{Parent}{$y$}
    \State $q \gets$ \Call{Right}{$y$}
    \State \Call{Key}{$x$} $\gets$ \Call{Key}{$y$}
    \State copy satellite data from $y$ to $x$
    \State replace $y$ with \Call{Right}{$y$}
    \State $x \gets y$
  \EndIf
  \If{\Call{Color}{$x$} = BLACK}
    \State $T \gets$ \textproc{Delete-Fix}($T$, \Call{Make-Black}{$p$, $q$}, $q$ = NIL?)
  \EndIf
  \State release $x$
  \State \Return $T$
\EndFunction
\end{algorithmic}

删除算法接受树的根节点$T$和待删除节点$x$。如果待删除节点存在一个为空的分支,我们可以将$x$“切下”,并用另一个分支$q$来替代$x$。否则,我们在$x$的右子树中找到最小的节点$y$,用$y$替换$x$。然后递归地将$y$“切下”。如果被删除的节点$x$的颜色为黑色,我们调用\textproc{Make-Black}{$p$, $q$},来保持黑色属性,以便进行下一步的修复。

\begin{algorithmic}[1]
\Function{Make-Black}{$p$, $q$}
  \If{$p$ = NIL $\land$ $q$ = NIL}
    \State \Return NIL \Comment{删除只有一个叶子节点的树后,变为空}
  \ElsIf{$q$ = NIL}
    \State $n \gets$ Doubly Black NIL
    \State \Call{Parent}{$n$} $\gets p$
    \State \Return $n$
  \Else
    \State \Return \Call{Blacken}{$q$}
  \EndIf
\EndFunction
\end{algorithmic}

如果传入\textproc{Make-Black}的参数$p$和$q$都为空,说明我们在删除只有一个叶子节点的树,删除后树变为空。否则如果父节点$p$不为空,而节点$q$为空。说明我们删除了一个黑色的叶子节点。这相当于,此时一个NIL节点替换了被删除的黑色节点。根据红黑树的性质3,NIL节点实际上都是黑色的。我们可以把这一NIL节点变成“双重黑色”NIL节点来保持其所在路径上的黑色节点数目不变。最后,如果$p$、$q$都不为空,我们调用\textproc{Blacken}检查$q$的颜色,如果是红色的,将它重新染成黑色,如果$q$已经是黑色的,我们将它染成双重黑色。

为了最终恢复红黑树的性质,我们需要通过树的旋转操作和重新染色,最终去掉“双重黑色”。这里有三种情况需要处理。每一种情况中,双重黑色的节点即可以是普通节点,也可以是双重黑色的空节点。我们首先看第一种情况。

\subsection{双重黑色节点的兄弟为黑色,并且该兄弟节点有一个红色子节点}
对于这种情况,我们可以通过旋转操作来修复。总共有四种不同的细分情况,它们全部可以变换到一种统一的形式。如图\ref{fig:del-case1}所示。

\begin{figure}[htbp]
   \centering
   \includegraphics[scale=0.4]{../../../datastruct/tree/red-black-tree/img/del-case1.eps}
   \caption{双重黑色节点的兄弟为黑色,并且该兄弟节点有一个红色子节点。这种情况可以通过一次旋转操作来修复。}
   \label{fig:del-case1}
\end{figure}

下面的算法描述了针对这一情况的处理。

\begin{algorithmic}[1]
\Function{Delete-Fix}{$T$, $x$, $f$}
  \State $n \gets$ NIL
  \If{$f$ = True}  \Comment{$x$是一个双重黑色NIL节点}
    \State $n \gets x$
  \EndIf
  \If{$x$ = NIL} \Comment{将只有一个叶子节点的树删空}
    \State \Return NIL
  \EndIf
  \While{$x \neq T \lor$ \Call{Color}{$x$} $= \mathcal{B}^2$}
    \Comment{$x$ isn't root or $x$ isn't doubly black}
    \If{\Call{Sibling}{$x$} $\neq$ NIL} \Comment{双重黑色节点的兄弟节点不为空}
        \State $s \gets$ \Call{Sibling}{$x$}
        \State ...
        \If{$s$ is black $\land$ \Call{Left}{$s$} is black}
          \Comment{兄弟为黑,且一个侄子为红}
          \If{$x = $ \textproc{Left}(\Call{Parent}{$x$})}
            \Comment{双重黑色节点是其父的左孩子}
            \State set $x$, \Call{Parent}{$x$}, and \Call{Left}{$s$} all black
            \State $T \gets$ \Call{Rotate-Right}{$T$, $s$}
            \State $T \gets$ \textproc{Rotate-Left}($T$, \Call{Parent}{$x$})
          \Else \Comment{双重黑色节点是其父的右孩子}
            \State set $x$, \Call{Parent}{$x$}, $s$, and \Call{Left}{$s$} all black
            \State $T \gets$ \textproc{Rotate-Right}($T$, \Call{Parent}{$x$})
          \EndIf
        \ElsIf{$s$ is black $\land$ \Call{Right}{$s$} is black}
          \If{$x = $ \textproc{Left}(\Call{Parent}{$x$})}
            \State set $x$, \Call{Parent}{$x$}, $s$, and \Call{Right}{$s$} all black
            \State $T \gets$ \textproc{Rotate-Left}($T$, \Call{Parent}{$x$})
          \Else
            \State set $x$, \Call{Parent}{$x$}, and \Call{Right}{$s$} all black
            \State $T \gets$ \Call{Rotate-Left}{$T$, $s$}
            \State $T \gets$ \textproc{Rotate-Right}($T$, \Call{Parent}{$x$})
          \EndIf
        \State ...
        \EndIf
    \EndIf
  \EndWhile
\EndFunction
\end{algorithmic}

\subsection{双重黑色节点的兄弟节点为红色}
这种情况下,我们可以通过旋转,将其变换为$p 1.1$和$p 1.2$。如图\ref{fig:del-case2}所示。

\begin{figure}[htbp]
  \centering
  \includegraphics[scale=0.4]{../../../datastruct/tree/red-black-tree/img/del-case3.eps}
  \caption{双重黑色节点的兄弟节点为红色} \label{fig:del-case2}
\end{figure}

在公式(\ref{eq:db-case-1a})的基础上增加这一处理可以得到公式(\ref{eq:db-case-2})。

\be
fixBlack^2(T) = \left \{
  \begin{array}
  {r@{\quad:\quad}l}
  ... & ... \\
  fixBlack^2(\mathcal{B}, fixBlack^2((\mathcal{R}, A, x, B), y, C) & p 2.1 \\
  fixBlack^2(\mathcal{B}, A, x, fixBlack^2((\mathcal{R}, B, y, C)) & p 2.2 \\
  T & otherwise
  \end{array}
\right .
\label{eq:db-case-2}
\ee

其中$p 2.1$和$p 2.2$表示如下:

\[
p 2.1 : \{ color(T) = \mathcal{B} \land color(T_l) = \mathcal{B}^2 \land color(T_r) = \mathcal{R} \}
\]

\[
p 2.2 : \{ color(T) = \mathcal{B} \land color(T_l) = \mathcal{R} \land color(T_r) = \mathcal{B}^2 \}
\]

\subsection{双重黑色节点的兄弟节点为黑色,该兄弟节点的两个子节点也全是黑色。}
这种情况下,我们可以将这个兄弟节点染成红色,将双重黑色变回黑色,然后将双重黑色属性向上传递一层到父节点。如图\ref{fig:del-case3}所示,有两种对称的情况。

\begin{figure}[htbp]
  \centering
  \setlength{\unitlength}{1cm}
  \begin{picture}(10, 4)
  \put(5, 2){$\Rightarrow$}
  \subfloat[$x$的颜色为红或者黑。]{\includegraphics[scale=0.4]{../../../datastruct/tree/red-black-tree/img/case2-a.ps}}
  \subfloat[若$x$此前的颜色为红,将其变为黑色,否则变为双重黑色。]{\includegraphics[scale=0.4]{../../../datastruct/tree/red-black-tree/img/case2-a1.ps}}
  \end{picture}
  \\
  \begin{picture}(10, 5)
  \put(5, 2){$\Rightarrow$}
  \subfloat[$y$的颜色为红或者黑。]{\includegraphics[scale=0.4]{../../../datastruct/tree/red-black-tree/img/case2-b.ps}}
  \subfloat[若$y$此前的颜色为红,将其变为黑色,否则变为双重黑色。]{\includegraphics[scale=0.4]{../../../datastruct/tree/red-black-tree/img/case2-b1.ps}}
  \end{picture}
  \\
  \begin{picture}(1, 0.5)\end{picture} %pad
  \caption{将双重黑色向上传递} \label{fig:del-case3}
\end{figure}

我们继续在式(\ref{eq:db-case-2})的基础上增加修复的定义。

\be
fixBlack^2(T) = \left \{
  \begin{array}
  {r@{\quad:\quad}l}
  ... & ... \\
  mkBlk((\mathcal{C}, mkBlk(A), x, (\mathcal{R}, B, y, C))) & p 3.1 \\
  mkBlk((\mathcal{C}, (\mathcal{R}, A, x, B), y, mkBlk(C))) & p 3.2 \\
  ... & ...
  \end{array}
\right .
\label{eq:db-case-3}
\ee

其中$p 3.1$和$p 3.2$定义如下:

\[
p 3.1 : \left \{ \begin{array}{l}
  T = (\mathcal{C}, A, x, (\mathcal{B}, B, y, C)) \land \\
  color(A) = \mathcal{B}^2 \land color(B) = color(C) = \mathcal{B}
  \end{array} \right \}
\]

\[
p 3.2 : \left \{ \begin{array}{l}
  T = (\mathcal{C}, (\mathcal{B}, A, x, B), y, C) \land \\
  color(C) = \mathcal{B}^2 \land color(A) = color(B) = \mathcal{B}
  \end{array} \right \}
\]

如果双重黑色的节点是双重黑色空节点$\Phi$,经过重新染色后,可以将其恢复为普通空节点。我们在式\ref{eq:db-case-3}的基础上增加双重黑色空节点的处理:

\be
fixBlack^2(T) = \left \{
  \begin{array}
  {r@{\quad:\quad}l}
  ... & ... \\
  mkBlk((\mathcal{C}, mkBlk(A), x, (\mathcal{R}, B, y, C))) & p 2.1 \\
  mkBlk((\mathcal{C}, \phi, x, (\mathcal{R}, B, y, C))) & p 2.1' \\
  mkBlk((\mathcal{C}, (\mathcal{R}, A, x, B), y, mkBlk(C))) & p 2.2 \\
  mkBlk((\mathcal{C}, (\mathcal{R}, A, x, B), y, \phi)) & p 2.2' \\
  ... & ...
  \end{array}
\right .
\label{eq:db-case-3a}
\ee

其中$p 3.1'$和$p 3.2'$定义如下:

\[
p 3.1' : \left \{ \begin{array}{l}
  T = (\mathcal{C}, \Phi, x, (\mathcal{B}, B, y, C)) \land \\
  color(B) = color(C) = \mathcal{B}
  \end{array} \right \}
\]

\[
p 3.2' : \left \{ \begin{array}{l}
  T = (\mathcal{C}, (\mathcal{B}, A, x, B), y, \Phi) \land \\
  color(A) = color(B) = \mathcal{B}
  \end{array} \right \}
\]

至此,我们对于双重黑色的全部情况都完成了修复。算法被定义为一个递归函数。它有两个终止条件:一个是$p1.1$和$p1.2$,双重黑色节点被直接消除了;另外一个是将双重黑色继续向上传递,直到根节点。由于算法最终会将根节点染成黑色,所以双重黑色也会被消除。

综合公式(\ref{eq:db-case-1a})、(\ref{eq:db-case-2})和(\ref{eq:db-case-3a}),我们可以得到最终的Haskell删除程序。

\begin{lstlisting}[style=Haskell]
-- 兄弟节点为黑色,并且有一个红色子节点
fixDB color a@(Node BB _ _ _) x (Node B (Node R b y c) z d)
      = Node color (Node B (makeBlack a) x b) y (Node B c z d)
fixDB color BBEmpty x (Node B (Node R b y c) z d)
      = Node color (Node B Empty x b) y (Node B c z d)
fixDB color a@(Node BB _ _ _) x (Node B b y (Node R c z d))
      = Node color (Node B (makeBlack a) x b) y (Node B c z d)
fixDB color BBEmpty x (Node B b y (Node R c z d))
      = Node color (Node B Empty x b) y (Node B c z d)
fixDB color (Node B a x (Node R b y c)) z d@(Node BB _ _ _)
      = Node color (Node B a x b) y (Node B c z (makeBlack d))
fixDB color (Node B a x (Node R b y c)) z BBEmpty
      = Node color (Node B a x b) y (Node B c z Empty)
fixDB color (Node B (Node R a x b) y c) z d@(Node BB _ _ _)
      = Node color (Node B a x b) y (Node B c z (makeBlack d))
fixDB color (Node B (Node R a x b) y c) z BBEmpty
      = Node color (Node B a x b) y (Node B c z Empty)
-- 兄弟节点是红色
fixDB B a@(Node BB _ _ _) x (Node R b y c) = fixDB B (fixDB R a x b) y c
fixDB B a@BBEmpty x (Node R b y c) = fixDB B (fixDB R a x b) y c
fixDB B (Node R a x b) y c@(Node BB _ _ _) = fixDB B a x (fixDB R b y c)
fixDB B (Node R a x b) y c@BBEmpty = fixDB B a x (fixDB R b y c)
-- 兄弟节点和它的两个子节点都是黑色,向上传递黑色
fixDB color a@(Node BB _ _ _) x (Node B b y c) = makeBlack (Node color (makeBlack a) x (Node R b y c))
fixDB color BBEmpty x (Node B b y c) = makeBlack (Node color Empty x (Node R b y c))
fixDB color (Node B a x b) y c@(Node BB _ _ _) = makeBlack (Node color (Node R a x b) y (makeBlack c))
fixDB color (Node B a x b) y BBEmpty = makeBlack (Node color (Node R a x b) y Empty)
-- 其他情况
fixDB color l k r = Node color l k r
\end{lstlisting}

对于含有$n$个节点的红黑树,删除算法的复杂度为$O(\lg n)$。

\begin{Exercise}

\begin{itemize}
\item 选用一种编程语言,实现本节提到的“标记――重建”删除算法:也就是先将要删除的节点标记,但不进行真正的删除。当被标记的节点数目超过50\%的时候,用全部未标记的节点重建树。
\item 为什么不需要在$mkBlk$的调用处,显示地再调用$fixBlack^2$?
\end{itemize}

\end{Exercise}

\ifx\wholebook\relax \else
\begin{thebibliography}{99}

\bibitem{CLRS}
Thomas H. Cormen, Charles E. Leiserson, Ronald L. Rivest and Clifford Stein.
``Introduction to Algorithms, Second Edition''. ISBN:0262032937. The MIT Press. 2001 (《算法导论》中文版)

\end{thebibliography}

\end{document}
\fi
