\ifx\wholebook\relax \else
% ------------------------

\documentclass[UTF8]{article}
%------------------- Other types of document example ------------------------
%
%\documentclass[twocolumn]{IEEEtran-new}
%\documentclass[12pt,twoside,draft]{IEEEtran}
%\documentstyle[9pt,twocolumn,technote,twoside]{IEEEtran}
%
%-----------------------------------------------------------------------------
%
% loading packages
%

\RequirePackage{ifpdf}
\RequirePackage{ifxetex}

%
%
\ifpdf
  \RequirePackage[pdftex,%
       bookmarksnumbered,%
              colorlinks,%
          linkcolor=blue,%
              hyperindex,%
        plainpages=false,%
       pdfstartview=FitH]{hyperref}
\else\ifxetex
  \RequirePackage[bookmarksnumbered,%
               colorlinks,%
           linkcolor=blue,%
               hyperindex,%
         plainpages=false,%
        pdfstartview=FitH]{hyperref}
\else
  \RequirePackage[dvipdfm,%
        bookmarksnumbered,%
               colorlinks,%
           linkcolor=blue,%
               hyperindex,%
         plainpages=false,%
        pdfstartview=FitH]{hyperref}
\fi\fi
%\usepackage{hyperref}

% other packages
%--------------------------------------------------------------------------
\usepackage{graphicx, color}
\usepackage{subfig}
\usepackage{multicol}
\usepackage{tikz}
\usetikzlibrary{matrix,positioning,shapes}

\usepackage{amsmath, amsthm, amssymb} % for math
\usepackage{exercise} % for exercise
\usepackage{import} % for nested input

%
% for programming
%
\usepackage{verbatim}
\usepackage{fancyvrb}
\usepackage{listings}
%\usepackage{algorithmic} %old version; we can use algorithmicx instead
%\usepackage[plain]{algorithm} %remove rule (horizontal line on top/below the algorithm
\usepackage{algorithm} %to remove rules change to \usepackage[plain]{algorithm}
%\usepackage{algorithm2e}
\usepackage[noend]{algpseudocode} %for pseudo code, include algorithmicsx automatically
\usepackage{appendix}
\usepackage{makeidx} % for index support
\usepackage{titlesec}

\usepackage[cm-default]{fontspec}
\usepackage{xunicode}
%\usepackage{fontenc}
\usepackage{textcomp}
\usepackage{url}

% detect and select Chinese font
% ------------------------------
% the following cmd can list all availabe Chinese fonts in host.
% fc-list :lang=zh
\def\myfont{STSong}  % Under Mac OS X
\def\linuxfallback{WenQuanYi Micro Hei} % Under Linux
\def\winfallback{SimSun} % Under Windows
\suppressfontnotfounderror1 % Avoid setting exit code (error level) to break make process
\count255=\interactionmode
\batchmode
\font\foo="\myfont"\space at 10pt
\ifx\foo\nullfont
  \font\foo = "\linuxfallback"\space at 10pt
  \ifx\foo\nullfont
    \font\foo = "\winfallback"\space at 10pt
    \ifx\foo\nullfont
      \errorstopmode
      \errmessage{no suitable Chinese font found}
    \else
      \let\myfont=\winfallback % Windows
    \fi
  \else
    \let\myfont=\linuxfallback % Linux
  \fi
\fi
\interactionmode=\count255
\setmainfont[Mapping=tex-text]{\myfont}
\setmonofont[Scale=MatchLowercase]{Monaco}   % 英文等宽字体

\XeTeXlinebreaklocale "zh"  % to solve the line breaking issue
\XeTeXlinebreakskip = 0pt plus 1pt minus 0.1pt

\titleformat{\paragraph}
{\normalfont\normalsize\bfseries}{\theparagraph}{1em}{}
\titlespacing*{\paragraph}
{0pt}{3.25ex plus 1ex minus .2ex}{1.5ex plus .2ex}

\lstdefinelanguage{Smalltalk}{
  morekeywords={self,super,true,false,nil,thisContext}, % This is overkill
  morestring=[d]',
  morecomment=[s]{"}{"},
  alsoletter={\#:},
  escapechar={!},
  literate=
    {BANG}{!}1
    {UNDERSCORE}{\_}1
    {\\st}{Smalltalk}9 % convenience -- in case \st occurs in code
    % {'}{{\textquotesingle}}1 % replaced by upquote=true in \lstset
    {_}{{$\leftarrow$}}1
    {>>>}{{\sep}}1
    {^}{{$\uparrow$}}1
    {~}{{$\sim$}}1
    {-}{{\sf -\hspace{-0.13em}-}}1  % the goal is to make - the same width as +
    %{+}{\raisebox{0.08ex}{+}}1		% and to raise + off the baseline to match -
    {-->}{{\quad$\longrightarrow$\quad}}3
	, % Don't forget the comma at the end!
  tabsize=2
}[keywords,comments,strings]

% for literate Haskell code
\lstdefinestyle{Haskell}{
  flexiblecolumns=false,
  basewidth={0.5em,0.45em},
  morecomment=[l]--,
  literate={+}{{$+$}}1 {/}{{$/$}}1 {*}{{$*$}}1 {=}{{$=$}}1
           {>}{{$>$}}1 {<}{{$<$}}1 {\\}{{$\lambda$}}1
           {\\\\}{{\char`\\\char`\\}}1
           {->}{{$\rightarrow$}}2 {>=}{{$\geq$}}2 {<-}{{$\leftarrow$}}2
           {<=}{{$\leq$}}2 {=>}{{$\Rightarrow$}}2
           {\ .}{{$\circ$}}2 {\ .\ }{{$\circ$}}2
           {>>}{{>>}}2 {>>=}{{>>=}}2
           {|}{{$\mid$}}1
}

\lstloadlanguages{C, C++, Lisp, Haskell, Python, Smalltalk}

\lstset{
  basicstyle=\small\ttfamily,
  commentstyle=\rmfamily,
  texcl=true,
  showstringspaces = false,
  upquote=true,
  flexiblecolumns=false
}

% ======================================================================

\def\BibTeX{{\rm B\kern-.05em{\sc i\kern-.025em b}\kern-.08em
    T\kern-.1667em\lower.7ex\hbox{E}\kern-.125emX}}

%
% mathematics
%
\newcommand{\be}{\begin{equation}}
\newcommand{\ee}{\end{equation}}
\newcommand{\bmat}[1]{\left( \begin{array}{#1} }
\newcommand{\emat}{\end{array} \right) }
\newcommand{\VEC}[1]{\mbox{\boldmath $#1$}}

% numbered equation array
\newcommand{\bea}{\begin{eqnarray}}
\newcommand{\eea}{\end{eqnarray}}

% equation array not numbered
\newcommand{\bean}{\begin{eqnarray*}}
\newcommand{\eean}{\end{eqnarray*}}

\newtheorem{theorem}{定理}[section]
\newtheorem{lemma}[theorem]{引理}
\newtheorem{proposition}[theorem]{Proposition}
\newtheorem{corollary}[theorem]{Corollary}

% 中文书籍设置
% ====================================
\renewcommand\contentsname{目\ 录}
%\renewcommand\listfigurename{插图目录}
%\renewcommand\listtablename{表格目录}
\renewcommand\figurename{图}
\renewcommand\tablename{表}
\renewcommand\proofname{证明}
\renewcommand\ExerciseName{练习}
%\renewcommand{\algorithmcfname}{算法}

\ifx\wholebook\relax
\renewcommand\bibname{参\ 考\ 文\ 献}                    %book类型
%\newtheorem{Definition}[Theorem]{定义}
\newtheorem{Theorem}{定理}[chapter]
\newtheorem{example}{例题}[chapter]
\else
\renewcommand\refname{参\ 考\ 文\ 献}
\fi

%\setcounter{secnumdepth}{4}
\titleformat{\chapter}
  {\normalfont\bfseries\Large}
  {第\arabic{chapter}章}
  {12pt}{\Large}
%% \titleformat{\subsection}
%%   {\normalfont\bfseries\large}
%%   {\CJKnumber{\arabic{subsection}}、}
%%   {12pt}{\large}
%% \titleformat{\subsubsection}
%%   {\normalfont\bfseries\normalsize}
%%   {\arabic{subsubsection}.}
%%   {12pt}{\normalsize}

%\renewcommand{\baselinestretch}{1.5}                        %文章行间距为1.5倍。

\makeatletter
\newcommand{\verbatimfont}[1]{\renewcommand{\verbatim@font}{\ttfamily#1}}
\makeatother

\setcounter{tocdepth}{4}
\setcounter{secnumdepth}{4}

%\verbatimfont{\footnotesize}


\setcounter{page}{1}

\begin{document}

%--------------------------

% ================================================================
%                 COVER PAGE
% ================================================================

\title{并不复杂的红黑树}

\author{刘新宇
\thanks{{\bfseries 刘新宇} \newline
  Email: liuxinyu95@gmail.com \newline}
  }

\maketitle
\fi

\markboth{红黑树}{初等算法}

\ifx\wholebook\relax
\chapter{并不复杂的红黑树}
\numberwithin{Exercise}{chapter}
\fi

% ================================================================
%                 Introduction
% ================================================================
上一章中,我们给过一个例子:通过使用二叉搜索树来统计文章中每个词出现的次数。

从这个例子出发,人们很自然希望用二叉搜索树处理电话黄页\footnote{一种公共发布的电话号码簿,因用黄色纸张印刷所以称为黄页。},并用它来查询某联系人的电话。

我们可以把之前的例子代码稍做改动来实现这一功能:

\begin{lstlisting}
int main(int, char** ) {
    ifstream f("yp.txt");
    map<string, string> dict;
    string name, phone;
    while(f>>name && f>>phone)
        dict[name]=phone;
    for(;;) {
        cout<<"\nname: ";
        cin>>name;
        if(dict.find(name) == dict.end())
            cout<<"not found";
        else
            cout<<"phone: "<<dict[name];
    }
}
\end{lstlisting}

这段程序运行良好。但是,如果我们把STL库提供的map换成普通的二叉搜索树,程序的性能就变差了。尤其是搜索诸如Zara、Zed、Zulu等姓名时特别明显。

电话黄页通常是按照字典字母顺序(lexicographic order)印刷的,因此姓名会按照升序排列。如果我们依次把数字1, 2, 3, ..., $n$插入二叉搜索树,就会得如图\ref{fig:unbalanced-tree}中的结果。

\begin{figure}[htbp]
    \centering
	\includegraphics[scale=0.5]{img/unbalanced.ps}
    \caption{不平衡的树} \label{fig:unbalanced-tree}
\end{figure}

这是一棵极不平衡的二叉树。对于高度为$h$的二叉搜索树,查找算法的复杂度为$O(h)$。如果树比较平衡,我们就能够达到$O(\lg n)$的性能。但在如图\ref{fig:unbalanced-tree}所示的极端情况下,查找的性能退化为$O(n)$。几乎和链表一样。

\begin{Exercise}

\begin{itemize}
\item 对于巨大的电话黄页列表,为了加快速度,一个想法是使用两个并发的任务(task,可以是线程或进程)。一个任务从头部向后读“姓名――电话”信息,另外一个任务从尾部向前读。当两个任务在中间相遇时程序结束。这样构建出的二叉搜索树是什么样子的?如果我们把黄页列表分割成更多片断,使用更多的任务会得到什么结果?
\item 参考图\ref{fig:unbalanced-trees},你能找到更多的情况造成二叉搜索树表现不佳么?
\end{itemize}

\end{Exercise}

\begin{figure}[htbp]
       \centering
       \subfloat[]{\includegraphics[scale=0.4]{img/unbalanced-2.ps}}
       \subfloat[]{\includegraphics[scale=0.4]{img/unbalanced-zigzag.ps}} \\
       \subfloat[]{\includegraphics[scale=0.4]{img/unbalanced-3.ps}}
       \caption{一些不平衡的二叉树}
       \label{fig:unbalanced-trees}
\end{figure}

\subsection{如何保证树的平衡}

为了避免出现不平衡情况,我们可以预先使用随机算法将输入序列打乱(参见\cite{CLRS}的12.4节)。但这个方法有一定的局限性,例如序列是由用户交互输入的,每当用户输入一个key后,树就会被更新。也就是说,树是随着用户输入逐步构建的。

人们已经发现了好几种方法可以解决二叉搜索树的平衡问题。它们中有很多依赖于二叉树的旋转(rotation)操作。旋转操作可以在保持元素顺序的前提下,改变树的结构。因此可以用来提高平衡性。

本章介绍红黑树,一种被广泛使用的自平衡二叉搜索树(self-adjusting balanced binary search tree)。下一章我们介绍另外一种自平衡树――AVL树。在后面第8章关于二叉堆的章节中,我们还会遇到一种有趣的树――splay树,它能够随着操作,逐渐把树变得越来越平衡。

\subsection{树的旋转}
\index{树的旋转}

\begin{figure}[htbp]
   \centering
   \setlength{\unitlength}{1cm}
   \begin{picture}(10, 4)
   \put(5, 2){$\Longleftrightarrow$}
   \subfloat[]{\includegraphics[scale=0.4]{img/rotate-r.ps}}
   \subfloat[]{\includegraphics[scale=0.4]{img/rotate-l.ps}}
   \end{picture}
   \\
   \begin{picture}(1, 0.5)\end{picture} %pad
   \caption{树的旋转,左旋操作将(a)中的树变换为(b);右旋操作是左旋的逆变换。}
   \label{fig:tree-rotation}
\end{figure}

树的旋转是一种特殊的操作,它在保持中序遍历结果不变的情况下,改变树的结构。这是因为存在多个不同的二叉搜索树对应到一个特定的中序遍历顺序。图\ref{fig:tree-rotation}描述了旋转操作。图中左侧的二叉搜索树,经过左旋可以变换为右侧的树,而右旋是左旋的逆变换。

旋转操作可以通过一系列的步骤来描述。我们也可以利用模式匹配(pattern matching),非常容易地定义它们。将非空的二叉树记为三元组$T=(T_l, k, T_r)$,下面的函数定义了左右旋转。

\be
rotateL(T) = \left \{
  \begin{array}
  {r@{\quad:\quad}l}
  ((a, X, b), Y, c) & T = (a, X, (b, Y, c)) \\
  T & otherwise
  \end{array}
\right .
\ee

\be
rotateR(T) = \left \{
  \begin{array}
  {r@{\quad:\quad}l}
  (a, X, (b, Y, c)) & T = ((a, X, b), Y, c)) \\
  T & otherwise
  \end{array}
\right .
\ee

用伪代码描述时,除了左右分支和key,还需要设置好父节点。

\begin{algorithmic}[1]
\Function{Left-Rotate}{$T, x$}
  \State $p \gets$ \Call{Parent}{$x$}
  \State $y \gets$ \Call{Right}{$x$} \Comment{假设$y \ne$ NIL}
  \State $a \gets$ \Call{Left}{$x$}
  \State $b \gets$ \Call{Left}{$y$}
  \State $c \gets$ \Call{Right}{$y$}
  \State \Call{Replace}{$x, y$}
  \State \Call{Set-Children}{$x, a, b$}
  \State \Call{Set-Children}{$y, x, c$}
  \If{$p = $ NIL}
    \State $T \gets y$
  \EndIf
  \State \Return $T$
\EndFunction

\Statex

\Function{Right-Rotate}{$T, y$}
  \State $p \gets$ \Call{Parent}{$y$}
  \State $x \gets$ \Call{Left}{$y$} \Comment{假设$x \ne$ NIL}
  \State $a \gets$ \Call{Left}{$x$}
  \State $b \gets$ \Call{Right}{$x$}
  \State $c \gets$ \Call{Right}{$y$}
  \State \Call{Replace}{$y, x$}
  \State \Call{Set-Children}{$y, b, c$}
  \State \Call{Set-Children}{$x, a, y$}
  \If{$p = $ NIL}
    \State $T \gets x$
  \EndIf
  \State \Return $T$
\EndFunction

\Statex

\Function{Set-Left}{$x, y$}
  \State \Call{Left}{$x$} $\gets y$
  \If{$y \ne$ NIL}
    \Call{Parent}{$y$} $\gets x$
  \EndIf
\EndFunction

\Statex

\Function{Set-Right}{$x, y$}
  \State \Call{Right}{$x$} $\gets y$
  \If{$y \ne$ NIL}
    \Call{Parent}{$y$} $\gets x$
  \EndIf
\EndFunction

\Statex

\Function{Set-Children}{$x, L, R$}
  \State \Call{Set-Left}{$x, L$}
  \State \Call{Set-Right}{$x, R$}
\EndFunction

\Statex

\Function{Replace}{$x, y$}
  \If{\Call{Parent}{$x$} =NIL}
    \If{$y \ne$ NIL}
      \Call{Parent}{$y$} $\gets$ NIL
    \EndIf
  \ElsIf{\textproc{Left}(\Call{Parent}{$x$}) $= x$}
    \State \textproc{Set-Left}(\Call{Parent}{$x$}, $y$)
  \Else
    \State \textproc{Set-Right}(\Call{Parent}{$x$}, $y$)
  \EndIf
  \State \Call{Parent}{$x$} $\gets$ NIL
\EndFunction
\end{algorithmic}

对比伪代码和模式匹配函数,后者主要从树结构变化的角度出发,而前者集中于描述变换的过程。如本章题目所讲,红黑树并不像看起来那样复杂。如果用传统方法来讲解红黑树,需要处理许多不同的情况。每种情况都必须仔细处理节点中的数据。如果换成函数式的思路,虽然会牺牲一些性能,但很多问题会变得简单直观。

本章中的大部份内容来自Chris Okasaki的成果\cite{okasaki}。

% ================================================================
% Definition
% ================================================================
\section{红黑树的定义}
\index{红黑树}

红黑树是一种自平衡二叉搜索树\cite{wiki-rbt}\footnote{红黑树是2-3-4树的等价形式(有关2-3-4树,可参考B树一章)。也就是说,对于任一2-3-4树,都存在至少一棵红黑树,使得它们中所有元素的顺序相同。}。通过对节点进行着色和旋转,红黑树可以很容易地保持树的平衡。

我们需要在二叉搜索树上增加一个额外的颜色信息。节点可以被涂成红色或黑色。如果一棵二叉搜索树满足下面的全部5条性质,我们称之为红黑树\cite{CLRS}。
\index{红黑树!红黑性质}

\begin{enumerate}
\item 任一节点要么是红色,要么是黑色。
\item 根节点为黑色。
\item 所有的叶节点(NIL节点)为黑色。
\item 如果一个节点为红色,则它的两个子节点都是黑色。
\item 对任一节点,从它出发到所有叶子节点的路径上包含相同数量的黑色节点。
\end{enumerate}

为什么这5条性质能保证红黑树的平衡性呢?因为它们有一个关键的特性:从根节点出发到达叶节点的所有路径中,最长路径不会超过最短路径两倍。

注意到第四条性质,它意味着不存在两个连续的红色节点。因此,最短的路径只含有黑色的节点,任何比最短路径长的路径上都分散有一些红色节点。根据性质五,从根节点出发的所有的路径都含有相同数量的黑色节点,这就最终保证了没有任何路径超过最短路径长度的两倍\cite{wiki-rbt}。图\ref{fig:rbt-example-with-nil}的例子展示了一棵红黑树。

\begin{figure}[htbp]
       \begin{center}
	\includegraphics[scale=0.5]{img/rbt-example-with-nil.ps}
        \caption{红黑树} \label{fig:rbt-example-with-nil}
       \end{center}
\end{figure}

由于所有的NIL节点都是黑色的,我们通常在图中将NIL节点隐藏不画出,如图\ref{fig:rbt-example-with-nil}所示。

\begin{figure}[htbp]
  \centering
  \includegraphics[scale=0.5]{img/rbt-example.ps}
  \caption{隐藏了所有NIL节点的红黑树} \label{fig:rbt-example}
\end{figure}

红黑树沿用所有二叉搜索树中不改变树结构的操作,包括查找、获取最大、最小值等。只有插入和删除操作是特殊的。

如前面单词统计的例子所示,有很多集合(set)和map容器是使用红黑树来实现的。包括C++标准库STL\cite{sgi-stl}。

由于只增加了一个颜色信息,我们可以复用二叉搜索树的节点定义。如下面的C++代码所示:

\lstset{language=C++}
\begin{lstlisting}
enum Color {Red, Black};

template <class T>
struct node {
    Color color;
    T key;
    node* left;
    node* right;
    node* parent;
};
\end{lstlisting}

对于函数式环境,我们可以在代数数据类型(algebraic data type)的定义中增加颜色参数,如下面的Haskell例子所示:

\lstset{language=Haskell}
\begin{lstlisting}[style=Haskell]
data Color = R | B
data RBTree a = Empty
              | Node Color (RBTree a) a (RBTree a)
\end{lstlisting}

\begin{Exercise}

\begin{itemize}
\item 利用红黑树的性质证明$n$个节点的红黑树的高度不会超过$2 \lg (n+1)$。
\end{itemize}

\end{Exercise}

% ================================================================
%                 Insertion
% ================================================================
\section{插入}
\index{红黑树!插入}

由于插入操作会改变树的结构,因此可能变得不平衡。为了保持红黑树的性质,我们需要在插入操作后进行变换来修复平衡问题。

当插入一个key时,我们可以把新节点一律染成红色。只要它不是根节点,除了第四条外的所有红黑树性质都可以满足。唯一的问题就是可能引入两个相邻的红色节点。

函数式和命令式实现使用不同的方法来修复平衡。前者直观简单,但是存在一点性能损失;后者有些复杂,但是具有更高的性能。大多数的算法书籍介绍后一种方法。本章中,我们关注函数式的方法,并展示这一方法极为简洁的特性。我们也会给出传统的命令式实现以作为对比。

Chris Okasaki指出,共有四种情况会违反红黑树的第四条性质。它们都带有两个相邻的红色节点。非常关键的一点是:它们可以被修复为一个统一形式\cite{okasaki},如图 \ref{fig:insert-fix}所示。

\begin{figure}[htbp]
  \centering
     \setlength{\unitlength}{1cm}
     \begin{picture}(15, 15)
        % arrows
        \put(4.5, 9.5){\vector(1, -1){1}}
        \put(4.5, 5){\vector(1, 1){1}}
        \put(10, 9.5){\vector(-1, -1){1}}
        \put(10, 5){\vector(-1, 1){1}}
        % graphics
	\put(0, 7){\includegraphics[scale=0.5]{img/insert-ll.ps}}
        \put(0, 0){\includegraphics[scale=0.5]{img/insert-lr.ps}}
        \put(7, 7){\includegraphics[scale=0.5]{img/insert-rr.ps}}
        \put(8.5, 0){\includegraphics[scale=0.5]{img/insert-rl.ps}}
        \put(2, 5){\includegraphics[scale=0.5]{img/insert-fixed.ps}}
      \end{picture}
     \caption{插入后需要修复的四种情况} \label{fig:insert-fix}
\end{figure}

注意到这一变换会把红色向上“移动”一层。如果进行自底向上的递归修复,最后一步会把根节点染成红色。根据红黑树的第二条性质,根节点必须是黑色的。因此我们最后要把根节点再染回黑色。为了方便,我们把节点的颜色记为$\mathcal{C}$,它有两个值,黑色$\mathcal{B}$和红色$\mathcal{R}$。这样一棵非空的红黑树可以表达为一个四元组$T=(\mathcal{C}, T_l, k, T_r)$。

\be
balance(T) = \left \{
  \begin{array}
  {r@{\quad:\quad}l}
  (\mathcal{R}, (\mathcal{B}, A, x, B), y, (\mathcal{B}, C, z, D)) & match(T) \\
  T & otherwise
  \end{array}
\right .
\ee

其中,函数$match(T)$用以判断树是否符合图\ref{fig:insert-fix}中四种需要修复的情况。定义如下:

\[
match(T) : \left \{ T = \begin{array}{l}
         (\mathcal{B}, (\mathcal{R}, (\mathcal{R}, A, x, B), y, C), z, D) \lor \\
         (\mathcal{B}, (\mathcal{R}, A, x, (\mathcal{R}, B, y, C), z, D)) \lor \\
         (\mathcal{B}, A, x, (\mathcal{R}, B, y, (\mathcal{R}, C, z, D))) \lor \\
         (\mathcal{B}, A, x, (\mathcal{R}, (\mathcal{R}, B, y, C), z, D)) \\
         \end{array} \right \}
\]

定义好函数$balance(T)$后,我们就可以修改二叉搜索树的插入函数,使其支持红黑树。

\be
insert(T, k) = makeBlack(ins(T, k))
\ee

其中$ins(T, k)$函数定义如下:

\be
ins(T, k) = \left \{
  \begin{array}
  {r@{\quad:\quad}l}
  (\mathcal{R}, \phi, k, \phi) & T = \phi \\
  balance((ins(T_l, k), k', T_r)) & k < k' \\
  balance((T_l, k', ins(T_r, k))) & otherwise
  \end{array}
\right.
\ee

如果待插入的树为空,则创建一个新的红色节点,节点的key就是待插入的$k$;否则,记树的左右分支和key分别为$T_l$、$T_r$和$k'$,我们比较$k$和$k'$的大小,递归地将它插入子分支中,然后再用$balance$函数恢复平衡。最后,我们使用$makeBlack(T)$函数把根节点染成黑色。

\be
makeBlack(T) = (\mathcal{B}, T_l, k, T_r)
\ee

在支持模式匹配的语言中,例如Haskell,插入算法可以实现为下面的程序:

\lstset{language=Haskell}
\begin{lstlisting}[style=Haskell]
insert t x = makeBlack $ ins t where
    ins Empty = Node R Empty x Empty
    ins (Node color l k r)
        | x < k     = balance color (ins l) k r
        | otherwise = balance color l k (ins r) --[3]
    makeBlack(Node _ l k r) = Node B l k r

balance B (Node R (Node R a x b) y c) z d =
                Node R (Node B a x b) y (Node B c z d)
balance B (Node R a x (Node R b y c)) z d =
                Node R (Node B a x b) y (Node B c z d)
balance B a x (Node R b y (Node R c z d)) =
                Node R (Node B a x b) y (Node B c z d)
balance B a x (Node R (Node R b y c) z d) =
                Node R (Node B a x b) y (Node B c z d)
balance color l k r = Node color l k r
\end{lstlisting} %$

程序中的\texttt{balance}函数略微有些不同,它的参数不是一棵树,而是节点的颜色、左侧分支、key和右侧分支。这样可以节省一对boxing和unboxing的操作。

我们没有处理存在重复key的情况。如果发生重复,我们可以选择覆盖,或者跳过不处理,还可以在节点中用一个链表存储重复的数据\cite{CLRS}。

图\ref{fig:insert-example}中给出了两个插入的例子。左侧是依次将11, 2, 14, 1, 7, 5, 8, 4插入的结果。右侧的是将序列1, 2, ..., 8插入的结果。可以看到,即使输入已序序列,红黑树仍然保持平衡。

\begin{figure}[htbp]
  \centering
  \includegraphics[scale=0.5]{img/insert-haskell.ps}
  \caption{根据不同的序列,使用插入算法产生的两棵红黑树} \label{fig:insert-example}
\end{figure}

通过将四种不同的情况统一变换成同一种形式,我们得到了一个极为简洁的算法。即使在不支持模式匹配的语言中,我们仍然可以编程检验树是否满足一定的模式。这样得到的结果和传统方法相比,仍然会简洁很多。读者可以参考本章附带的Lisp方言Scheme的例子代码。

对于含有$n$个节点的红黑树,这一插入算法的复杂度为$O(\lg n)$。

\begin{Exercise}

\begin{itemize}
\item 使用一种命令式语言,例如C、C++或Python来实现本节介绍的插入算法。注意:如果语言本身不支持模式匹配,需要编程检查四种不同的情况。
\end{itemize}

\end{Exercise}

% ================================================================
%                 Deletion
% ================================================================

\section{删除}
\index{红黑树!删除}

上一章曾指出,二叉搜索树的删除操作只在命令式环境下才有意义。这一结论同样适用于红黑树。大多数情况下,通常一次性将树构建好,然后频繁地进行查找。Okasaki曾经解释过为什么他没有给出红黑树的删除算法\cite{okasaki-blog}。其中一个原因就是删除比插入要麻烦得多。

我们希望通过本节的介绍,使读者了解到,在纯函数式的环境中,删除算法是可以实现的。纯函数式数据结构决定了树不会真的被改变,我们实际上是重建了一棵树\footnote{大多数函数式编程环境通过使用一种名为persistent的技术,可以复用树中没有改变的部份,从而减小重建的开销。}。实际应用中,往往是由用户(也就是编程者)来决定使用那种具体的方案。例如,我们可以不做任何操作,而仅仅标记一下要删除的节点。当带有删除标记的节点超过50\%的时候,用所有未标记的节点重建一棵树。

不仅是函数式环境,命令式的删除算法也比插入算法要复杂。这主要是因为在删除时,我们面临更多的情况需要修复平衡。删除也会破坏红黑树的性质,因此需要后继的处理以恢复平衡。

本节介绍的删除算法主要来自\cite{lyn}。只有在删除一个黑色的节点时才会引发问题。因为这样会破坏红黑树的第五条性质,使得某一路径上的黑色节点数目少于其他的路径。

在删除一个黑色节点时,我们可以通过引入“双重黑色”\cite{CLRS}的概念来恢复第五条性质。也就是说,虽然节点被删除了,我们把它的黑色保存在它的父节点中。如果父节点是红色的,我们将其变为黑色;但如果父节点已经是黑色的,它就会变成一个“双重黑色”的节点。

为了使用双重黑色概念,我们需要修改一下红黑树的定义,如下面的Haskell示例代码:

\lstset{language=Haskell}
\begin{lstlisting}[style=Haskell]
data Color = R | B | BB -- BB: 用于删除操作的双重黑色
data RBTree a = Empty | BBEmpty -- 双重黑色的空节点
              | Node Color (RBTree a) a (RBTree a)
\end{lstlisting}

删除一个节点时,我们先调用普通二叉搜索树的删除算法。如果被删除节点是黑色的,我们接下来进行修复。删除函数定义如下:

\be
delete(T, k) = blackenRoot(del(T, k))
\ee

其中

\be
del(T, k) = \left \{
  \begin{array}
  {r@{\quad:\quad}l}
  \phi & T = \phi \\
  fixBlack^2((\mathcal{C}, del(T_l, k), k', T_r)) & k < k' \\
  fixBlack^2((\mathcal{C}, T_l, k', del(T_r, k))) & k > k' \\
  \left \{
    \begin{array}{r@{\quad:\quad}l}
    mkBlk(T_r) & \mathcal{C} = \mathcal{B} \\
    T_r & otherwise
    \end{array}
  \right. & T_l = \phi \\
  \left \{
    \begin{array}{r@{\quad:\quad}l}
    mkBlk(T_l) & \mathcal{C} = \mathcal{B} \\
    T_l & otherwise
    \end{array}
  \right.  & T_r = \phi \\
  fixBlack^2((\mathcal{C}, T_l, k'', del(T_r, k''))) & otherwise
  \end{array}
\right.
\ee

函数$del$定义了各种不同的情况。如果树为空,这种边界情况下的删除结果也为空$\phi$;否则,如果待删除的key比当前节点的key小,我们递归地在左侧分支进行删除;如果比当前节点的key大,则递归地在右侧分支删除。由于可能引入双重黑色,所以需要后继的修复处理。

如果待删除的key恰好等于当前节点的key,我们需要将其“切下”。若当前节点有一个子分支为空,我们只要用另外一个分支替换当前节点,并且保持当前节点的颜色属性就可以了。否则,如果两个分支都不为空,我们从右侧子分支中找到最小值$k''=min(T_r)$,将其“切下”并替换掉当前的节点的key。

最后,函数$delete$通过调用$blackenRoot$强制将根节点染成黑色。

\be
blackenRoot(T) = \left \{
  \begin{array}
  {r@{\quad:\quad}l}
  \phi & T = \phi \\
  (\mathcal{B}, T_l, k, T_r) & otherwise \\
  \end{array}
\right .
\ee

和红黑树插入算法中的$makeBlack$函数相比,$blackenRoot$仅仅多了对为空树的处理。这一处理仅在删除时才需要考虑,因为插入操作不可能产生一棵空树,而删除操作是有可能的。

函数$mkBlk$用于保持被“切下”节点的黑色属性。如果被“切下”的节点为黑色,它将红色的结果改为黑色,把黑色的结果改为双重黑色。如果结果为空,则将其标记为双重黑色的空,记为$\Phi$。

\be
mkBlk(T) = \left \{
  \begin{array}
  {r@{\quad:\quad}l}
  \Phi & T = \phi \\
  (\mathcal{B}, T_l, k, T_r) & \mathcal{C} = \mathcal{R} \\
  (\mathcal{B}^2, T_l, k, T_r) & \mathcal{C} = \mathcal{B} \\
  T & otherwise
  \end{array}
\right .
\ee

其中,符号$\mathcal{B}^2$表示双重黑色。

将目前的函数定义翻译为Haskell可以得到下面的例子程序。

\begin{lstlisting}[style=Haskell]
delete t x = blackenRoot(del t x) where
    del Empty _ = Empty
    del (Node color l k r) x
        | x < k = fixDB color (del l x) k r
        | x > k = fixDB color l k (del r x)
        -- x == k, 删除此节点
        | isEmpty l = if color==B then makeBlack r else r
        | isEmpty r = if color==B then makeBlack l else l
        | otherwise = fixDB color l k' (del r k') where k'= min r
    blackenRoot (Node _ l k r) = Node B l k r
    blackenRoot _ = Empty

makeBlack (Node B l k r) = Node BB l k r -- 双重黑色
makeBlack (Node _ l k r) = Node B l k r
makeBlack Empty = BBEmpty
makeBlack t = t
\end{lstlisting}

删除算法中,唯一还没有定义的函数是$fixBlack^2$。我们需要在这个函数中,通过树的旋转操作和重新染色,最终去掉“双重黑色”。这里有三种情况需要处理。每一种情况中,双重黑色的节点即可以是普通节点,也可以是双重黑色的空节点$\Phi$。我们首先看第一种情况。

\subsection{双重黑色节点的兄弟为黑色,并且该兄弟节点有一个红色子节点}
对于这种情况,我们可以通过旋转操作来修复。总共有四种不同的细分情况,它们全部可以变换到一种统一的形式。如图\ref{fig:del-case1}所示。

\begin{figure}[htbp]
   \centering
   \includegraphics[scale=0.4]{img/del-case1.eps}
   \caption{双重黑色节点的兄弟为黑色,并且该兄弟节点有一个红色子节点。这种情况可以通过一次旋转操作来修复。}
   \label{fig:del-case1}
\end{figure}

我们通过模式匹配来处理这四种细分情况:

\be
fixBlack^2(T) = \left \{
  \begin{array}
  {r@{\quad:\quad}l}
  (\mathcal{C}, (\mathcal{B}, mkBlk(A), x, B), y, (\mathcal{B}, C, z, D)) & p 1.1 \\
  (\mathcal{C}, (\mathcal{B}, A, x, B), y, (\mathcal{B}, C, z, mkBlk(D))) & p 1.2 \\
  \end{array}
\right .
\label{eq:db-case-1}
\ee

其中$p 1.1$和$p 1.2$各代表两种细分情况:

\[
p 1.1 : \left \{ \begin{array}{l}
  T = (\mathcal{C}, A, x, (\mathcal{B}, (\mathcal{R}, B, y, C), z, D)) \land color(A) = \mathcal{B}^2 \\
  \lor \\
  T = (\mathcal{C}, A, x, (\mathcal{B}, B, y, (\mathcal{R}, C, z, D))) \land color(A) = \mathcal{B}^2
  \end{array} \right \}
\]

\[
p 1.2 : \left \{ \begin{array}{l}
  T = (\mathcal{C}, (\mathcal{B}, A, x, (\mathcal{R}, B, y, C)), z, D) \land color(D) = \mathcal{B}^2 \\
  \lor \\
  T = (\mathcal{C}, (\mathcal{B}, (\mathcal{R}, A, x, B), y, C), z, D) \land color(D) = \mathcal{B}^2
  \end{array} \right \}
\]

如果双重黑色的节点是双重黑色空节点$\Phi$,经过这样的旋转操作后,可以将其恢复为普通空节点。为此我们为式\ref{eq:db-case-1}增加双重黑色空节点的处理:

\be
fixBlack^2(T) = \left \{
  \begin{array}
  {r@{\quad:\quad}l}
  (\mathcal{C}, (\mathcal{B}, mkBlk(A), x, B), y, (\mathcal{B}, C, z, D)) & p 1.1 \\
  (\mathcal{C}, (\mathcal{B}, \phi, x, B), y, (\mathcal{B}, C, z, D)) & p 1.1' \\
  (\mathcal{C}, (\mathcal{B}, A, x, B), y, (\mathcal{B}, C, z, mkBlk(D))) & p 1.2 \\
  (\mathcal{C}, (\mathcal{B}, A, x, B), y, (\mathcal{B}, C, z, \phi)) & p 1.2' \\
  \end{array}
\right .
\label{eq:db-case-1a}
\ee

其中$p 1.1'$和$p 1.2'$的模式定义如下:

\[
p 1.1' : \left \{ \begin{array}{l}
  T = (\mathcal{C}, \Phi, x, (\mathcal{B}, (\mathcal{R}, B, y, C), z, D)) \\
  \lor \\
  T = (\mathcal{C}, \Phi, x, (\mathcal{B}, B, y, (\mathcal{R}, C, z, D)))
  \end{array} \right \}
\]

\[
p 1.2' : \left \{ \begin{array}{l}
  T = (\mathcal{C}, (\mathcal{B}, A, x, (\mathcal{R}, B, y, C)), z, \Phi) \\
  \lor \\
  T = (\mathcal{C}, (\mathcal{B}, (\mathcal{R}, A, x, B), y, C), z, \Phi)
  \end{array} \right \}
\]

\subsection{双重黑色节点的兄弟节点为红色}
这种情况下,我们可以通过旋转,将其变换为$p 1.1$和$p 1.2$。如图\ref{fig:del-case2}所示。

\begin{figure}[htbp]
  \centering
  \includegraphics[scale=0.4]{img/del-case3.eps}
  \caption{双重黑色节点的兄弟节点为红色} \label{fig:del-case2}
\end{figure}

在公式(\ref{eq:db-case-1a})的基础上增加这一处理可以得到公式(\ref{eq:db-case-2})。

\be
fixBlack^2(T) = \left \{
  \begin{array}
  {r@{\quad:\quad}l}
  ... & ... \\
  fixBlack^2(\mathcal{B}, fixBlack^2((\mathcal{R}, A, x, B), y, C) & p 2.1 \\
  fixBlack^2(\mathcal{B}, A, x, fixBlack^2((\mathcal{R}, B, y, C)) & p 2.2 \\
  T & otherwise
  \end{array}
\right .
\label{eq:db-case-2}
\ee

其中$p 2.1$和$p 2.2$表示如下:

\[
p 2.1 : \{ color(T) = \mathcal{B} \land color(T_l) = \mathcal{B}^2 \land color(T_r) = \mathcal{R} \}
\]

\[
p 2.2 : \{ color(T) = \mathcal{B} \land color(T_l) = \mathcal{R} \land color(T_r) = \mathcal{B}^2 \}
\]

\subsection{双重黑色节点的兄弟节点为黑色,该兄弟节点的两个子节点也全是黑色。}
这种情况下,我们可以将这个兄弟节点染成红色,将双重黑色变回黑色,然后将双重黑色属性向上传递一层到父节点。如图\ref{fig:del-case3}所示,有两种对称的情况。

\begin{figure}[htbp]
  \centering
  \setlength{\unitlength}{1cm}
  \begin{picture}(10, 4)
  \put(5, 2){$\Rightarrow$}
  \subfloat[$x$的颜色为红或者黑。]{\includegraphics[scale=0.4]{img/case2-a.ps}}
  \subfloat[若$x$此前的颜色为红,将其变为黑色,否则变为双重黑色。]{\includegraphics[scale=0.4]{img/case2-a1.ps}}
  \end{picture}
  \\
  \begin{picture}(10, 5)
  \put(5, 2){$\Rightarrow$}
  \subfloat[$y$的颜色为红或者黑。]{\includegraphics[scale=0.4]{img/case2-b.ps}}
  \subfloat[若$y$此前的颜色为红,将其变为黑色,否则变为双重黑色。]{\includegraphics[scale=0.4]{img/case2-b1.ps}}
  \end{picture}
  \\
  \begin{picture}(1, 0.5)\end{picture} %pad
  \caption{将双重黑色向上传递} \label{fig:del-case3}
\end{figure}

我们继续在式(\ref{eq:db-case-2})的基础上增加修复的定义。

\be
fixBlack^2(T) = \left \{
  \begin{array}
  {r@{\quad:\quad}l}
  ... & ... \\
  mkBlk((\mathcal{C}, mkBlk(A), x, (\mathcal{R}, B, y, C))) & p 3.1 \\
  mkBlk((\mathcal{C}, (\mathcal{R}, A, x, B), y, mkBlk(C))) & p 3.2 \\
  ... & ...
  \end{array}
\right .
\label{eq:db-case-3}
\ee

其中$p 3.1$和$p 3.2$定义如下:

\[
p 3.1 : \left \{ \begin{array}{l}
  T = (\mathcal{C}, A, x, (\mathcal{B}, B, y, C)) \land \\
  color(A) = \mathcal{B}^2 \land color(B) = color(C) = \mathcal{B}
  \end{array} \right \}
\]

\[
p 3.2 : \left \{ \begin{array}{l}
  T = (\mathcal{C}, (\mathcal{B}, A, x, B), y, C) \land \\
  color(C) = \mathcal{B}^2 \land color(A) = color(B) = \mathcal{B}
  \end{array} \right \}
\]

如果双重黑色的节点是双重黑色空节点$\Phi$,经过重新染色后,可以将其恢复为普通空节点。我们在式\ref{eq:db-case-3}的基础上增加双重黑色空节点的处理:

\be
fixBlack^2(T) = \left \{
  \begin{array}
  {r@{\quad:\quad}l}
  ... & ... \\
  mkBlk((\mathcal{C}, mkBlk(A), x, (\mathcal{R}, B, y, C))) & p 2.1 \\
  mkBlk((\mathcal{C}, \phi, x, (\mathcal{R}, B, y, C))) & p 2.1' \\
  mkBlk((\mathcal{C}, (\mathcal{R}, A, x, B), y, mkBlk(C))) & p 2.2 \\
  mkBlk((\mathcal{C}, (\mathcal{R}, A, x, B), y, \phi)) & p 2.2' \\
  ... & ...
  \end{array}
\right .
\label{eq:db-case-3a}
\ee

其中$p 3.1'$和$p 3.2'$定义如下:

\[
p 3.1' : \left \{ \begin{array}{l}
  T = (\mathcal{C}, \Phi, x, (\mathcal{B}, B, y, C)) \land \\
  color(B) = color(C) = \mathcal{B}
  \end{array} \right \}
\]

\[
p 3.2' : \left \{ \begin{array}{l}
  T = (\mathcal{C}, (\mathcal{B}, A, x, B), y, \Phi) \land \\
  color(A) = color(B) = \mathcal{B}
  \end{array} \right \}
\]

至此,我们对于双重黑色的全部情况都完成了修复。算法被定义为一个递归函数。它有两个终止条件:一个是$p1.1$和$p1.2$,双重黑色节点被直接消除了;另外一个是将双重黑色继续向上传递,直到根节点。由于算法最终会将根节点染成黑色,所以双重黑色也会被消除。

综合公式(\ref{eq:db-case-1a})、(\ref{eq:db-case-2})和(\ref{eq:db-case-3a}),我们可以得到最终的Haskell删除程序。

\begin{lstlisting}[style=Haskell]
-- 兄弟节点为黑色,并且有一个红色子节点
fixDB color a@(Node BB _ _ _) x (Node B (Node R b y c) z d)
      = Node color (Node B (makeBlack a) x b) y (Node B c z d)
fixDB color BBEmpty x (Node B (Node R b y c) z d)
      = Node color (Node B Empty x b) y (Node B c z d)
fixDB color a@(Node BB _ _ _) x (Node B b y (Node R c z d))
      = Node color (Node B (makeBlack a) x b) y (Node B c z d)
fixDB color BBEmpty x (Node B b y (Node R c z d))
      = Node color (Node B Empty x b) y (Node B c z d)
fixDB color (Node B a x (Node R b y c)) z d@(Node BB _ _ _)
      = Node color (Node B a x b) y (Node B c z (makeBlack d))
fixDB color (Node B a x (Node R b y c)) z BBEmpty
      = Node color (Node B a x b) y (Node B c z Empty)
fixDB color (Node B (Node R a x b) y c) z d@(Node BB _ _ _)
      = Node color (Node B a x b) y (Node B c z (makeBlack d))
fixDB color (Node B (Node R a x b) y c) z BBEmpty
      = Node color (Node B a x b) y (Node B c z Empty)
-- 兄弟节点是红色
fixDB B a@(Node BB _ _ _) x (Node R b y c) = fixDB B (fixDB R a x b) y c
fixDB B a@BBEmpty x (Node R b y c) = fixDB B (fixDB R a x b) y c
fixDB B (Node R a x b) y c@(Node BB _ _ _) = fixDB B a x (fixDB R b y c)
fixDB B (Node R a x b) y c@BBEmpty = fixDB B a x (fixDB R b y c)
-- 兄弟节点和它的两个子节点都是黑色,向上传递黑色
fixDB color a@(Node BB _ _ _) x (Node B b y c) = makeBlack (Node color (makeBlack a) x (Node R b y c))
fixDB color BBEmpty x (Node B b y c) = makeBlack (Node color Empty x (Node R b y c))
fixDB color (Node B a x b) y c@(Node BB _ _ _) = makeBlack (Node color (Node R a x b) y (makeBlack c))
fixDB color (Node B a x b) y BBEmpty = makeBlack (Node color (Node R a x b) y Empty)
-- 其他情况
fixDB color l k r = Node color l k r
\end{lstlisting}

对于含有$n$个节点的红黑树,删除算法的复杂度为$O(\lg n)$。

\begin{Exercise}

\begin{itemize}
\item 选用一种编程语言,实现本节提到的“标记――重建”删除算法:也就是先将要删除的节点标记,但不进行真正的删除。当被标记的节点数目超过50\%的时候,用全部未标记的节点重建树。
\item 为什么不需要在$mkBlk$的调用处,显示地再调用$fixBlack^2$?
\end{itemize}

\end{Exercise}

\section{命令式的红黑树算法$\star$}
\index{红黑树!命令式插入}

通过归纳,我们能够简洁地实现红黑树算法。作为对比,我们来看一下传统的命令式方法。

插入算法的基本思想仍然和二叉搜索树相同。此外,算法需要通过树的旋转操作修复平衡。

\begin{algorithmic}[1]
\Function{Insert}{$T, k$}
  \State $root \gets T$
  \State $x \gets$ \Call{Create-Leaf}{$k$}
  \State \Call{Color}{$x$} $\gets$ RED
  \State $p \gets$ NIL
  \While{$T \neq$ NIL}
    \State $p \gets T$
    \If{$k <$ \Call{Key}{$T$}}
      \State $T \gets $ \Call{Left}{$T$}
    \Else
      \State $T \gets $ \Call{Right}{$T$}
    \EndIf
  \EndWhile
  \State \Call{Parent}{$x$} $\gets p$
  \If{$p =$ NIL} \Comment{树$T$为空}
    \State \Return $x$
  \ElsIf{$k <$ \Call{Key}{$p$}}
    \State \Call{Left}{$p$} $\gets x$
  \Else
    \State \Call{Right}{$p$} $\gets x$
  \EndIf
  \State \Return \Call{Insert-Fix}{$root, x$}
\EndFunction
\end{algorithmic}

当插入一个新节点时,我们将其染成红色,然后修复平衡并返回。上述算法可以转换为下面的Python例子程序。

\lstset{language=Python}
\begin{lstlisting}
def rb_insert(t, key):
    root = t
    x = Node(key)
    parent = None
    while(t):
        parent = t
        if(key < t.key):
            t = t.left
        else:
            t = t.right
    if parent is None: #树为空
        root = x
    elif key < parent.key:
        parent.set_left(x)
    else:
        parent.set_right(x)
    return rb_insert_fix(root, x)
\end{lstlisting}

总共有3种基本情况需要修复。如果考虑左右对称,则需要修复6种情况。3种基本情况中的两种可以合并。新插入节点的父节点,以及父节点的兄弟节点均为红色。我们可以把它们变为黑色,然后把新插入节点的祖父节点染为红色。修复算法的实现如下:

\begin{algorithmic}[1]
\Function{Insert-Fix}{$T, x$}
  \While{\Call{Parent}{$x$} $\neq$ NIL $\land$ \textproc{Color}(\Call{Parent}{$x$}) = RED}
    \If{\textproc{Color}(\Call{Uncle}{$x$}) $=$ RED}
      \Comment{情况1:$x$的叔父节点是红色}
      \State \textproc{Color}(\Call{Parent}{$x$}) $\gets$ BLACK
      \State \textproc{Color}(\Call{Grand-Parent}{$x$}) $\gets$ RED
      \State \textproc{Color}(\Call{Uncle}{$x$}) $\gets$ BLACK
      \State $x \gets$ \Call{Grand-Parent}{$x$}
    \Else
      \Comment{$x$的叔父节点是黑色}
      \If{\Call{Parent}{$x$} = \textproc{Left}(\Call{Grand-Parent}{$x$})}
        \If{ $x =$ \textproc{Right}(\Call{Parent}{$x$})}
          \Comment{情况2:$x$是右侧子节点}
          \State $x \gets$ \Call{Parent}{$x$}
          \State $T \gets$ \Call{Left-Rotate}{$T, x$}
        \EndIf
        \Comment{情况3:$x$是左侧子节点}
        \State \textproc{Color}(\Call{Parent}{$x$}) $\gets$ BLACK
        \State \textproc{Color}(\Call{Grand-Parent}{$x$}) $\gets$ RED
        \State $T \gets$ \textproc{Right-Rotate}($T$, \Call{Grand-Parent}{$x$})
      \Else
         \If{ $x =$ \textproc{Left}(\Call{Parent}{$x$})}
          \Comment{情况2的对称情况}
          \State $x \gets$ \Call{Parent}{$x$}
          \State $T \gets$ \Call{Right-Rotate}{$T, x$}
        \EndIf
        \Comment{情况3的对称情况}
        \State \textproc{Color}(\Call{Parent}{$x$}) $\gets$ BLACK
        \State \textproc{Color}(\Call{Grand-Parent}{$x$}) $\gets$ RED
        \State $T \gets$ \textproc{Left-Rotate}($T$, \Call{Grand-Parent}{$x$})
      \EndIf
    \EndIf
  \EndWhile
  \State \Call{Color}{$T$} $\gets$ BLACK
  \State \Return $T$
\EndFunction
\end{algorithmic}

这一算法向红黑树中插入key的复杂度为$O(\lg n)$。和前面定义的$balance$函数相比,我们可以发现它们的差异。两种方法不仅仅在篇幅长短上不同,具体的逻辑也不一样。即使输入同一序列的key,两种方法也会构造出不同的红黑树。并且,和这一命令式算法相比,前面使用模式匹配的函数式算法存在一些性能上的损失。Okasaki在\cite{okasaki}中给出了关于函数式红黑树插入算法性能的详细分析。

上述修复算法可以实现为如下的Python例子程序。

\lstset{language=Python}
\begin{lstlisting}
# 修复连续的红色节点
def rb_insert_fix(t, x):
    while(x.parent and x.parent.color==RED):
        if x.uncle().color == RED:
            # 情况1: ((a:R x:R b) y:B c:R) ==> ((a:R x:B b) y:R c:B)
            set_color([x.parent, x.grandparent(), x.uncle()],
                      [BLACK, RED, BLACK])
            x = x.grandparent()
        else:
            if x.parent == x.grandparent().left:
                if x == x.parent.right:
                    # 情况2: ((a x:R b:R) y:B c) ==> 情况3
                    x = x.parent
                    t=left_rotate(t, x)
                # 情况3: ((a:R x:R b) y:B c) ==> (a:R x:B (b y:R c))
                set_color([x.parent, x.grandparent()], [BLACK, RED])
                t=right_rotate(t, x.grandparent())
            else:
                if x == x.parent.left:
                    # 情况2': (a x:B (b:R y:R c)) ==> 情况3'
                    x = x.parent
                    t = right_rotate(t, x)
                # 情况3': (a x:B (b y:R c:R)) ==> ((a x:R b) y:B c:R)
                set_color([x.parent, x.grandparent()], [BLACK, RED])
                t=left_rotate(t, x.grandparent())
    t.color = BLACK
    return t
\end{lstlisting}

图\ref{fig:imperative-insert}给出了两棵红黑树,它们是使用和图\ref{fig:insert-example}中完全相同的序列构造出的。我们可以发现它们明显不同。

\begin{figure}[htbp]
   \centering
   \subfloat[]{\includegraphics[scale=0.4]{img/clrs-fig-13-4.ps}}
   \subfloat[]{\includegraphics[scale=0.4]{img/python-insert.ps}}
   \caption{命令式算法构建出的红黑树}
   \label{fig:imperative-insert}
\end{figure}

红黑树的命令式删除算法和插入算法相比更为复杂。我们不在正文给出,读者可以参考本书附录B了解其详细实现。

\section{其它}
红黑树是最广泛使用的一种平衡二叉搜索树。另外一种自平衡二叉树是AVL树,我们将在下一章介绍。红黑树可以帮助我们了解其它更复杂的数据结构。如果我们将子节点的数目从两个扩展到$k$个,并且保持树的平衡,就可以演化到B树。如果我们在边上,而非在节点中存储数据,我们就得到了Trie。由于常见红黑树算法需要处理很多情况,代码篇幅较长,初学者往往会感觉红黑树很复杂。

Okasaki的工作使得红黑树算法变得容易理解。这激发了很多其它程序设计语言进行类似的实现\cite{rosetta}。本书中给出的Splay树、AVL树等模式匹配算法也是受到这一启发而完成的。

\ifx\wholebook\relax \else
\begin{thebibliography}{99}

\bibitem{CLRS}
Thomas H. Cormen, Charles E. Leiserson, Ronald L. Rivest and Clifford Stein.
``Introduction to Algorithms, Second Edition''. ISBN:0262032937. The MIT Press. 2001 (《算法导论》中文版)

\bibitem{okasaki}
Chris Okasaki. ``FUNCTIONAL PEARLS Red-Black Trees in a Functional Setting''. J. Functional Programming. 1998

\bibitem{okasaki-blog}
Chris Okasaki. ``Ten Years of Purely Functional Data Structures''. http://okasaki.blogspot.com/2008/02/ten-years-of-purely-functional-data.html

\bibitem{wiki-rbt}
Wikipedia. ``Red-black tree''. http://en.wikipedia.org/wiki/Red-black\_tree

\bibitem{lyn}
Lyn Turbak. ``Red-Black Trees''. http://cs.wellesley.edu/~cs231/fall01/red-black.pdf Nov. 2, 2001.

\bibitem{sgi-stl}
SGI STL. http://www.sgi.com/tech/stl/

\bibitem{rosetta}
Pattern matching. http://rosettacode.org/wiki/Pattern\_matching

\end{thebibliography}

\end{document}
\fi
