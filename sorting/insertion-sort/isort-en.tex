\ifx\wholebook\relax \else
% ------------------------

\documentclass{article}
%------------------- Other types of document example ------------------------
%
%\documentclass[twocolumn]{IEEEtran-new}
%\documentclass[12pt,twoside,draft]{IEEEtran}
%\documentstyle[9pt,twocolumn,technote,twoside]{IEEEtran}
%
%-----------------------------------------------------------------------------
%\input{../../../common.tex}
%
% loading packages
%

\RequirePackage{ifpdf}
\RequirePackage{ifxetex}

%
%
\ifpdf
  \RequirePackage[pdftex,%
       bookmarksnumbered,%
              colorlinks,%
          linkcolor=blue,%
              hyperindex,%
        plainpages=false,%
       pdfstartview=FitH]{hyperref}
\else\ifxetex
  \RequirePackage[bookmarksnumbered,%
               colorlinks,%
           linkcolor=blue,%
               hyperindex,%
         plainpages=false,%
        pdfstartview=FitH]{hyperref}
\else
  \RequirePackage[dvipdfm,%
        bookmarksnumbered,%
               colorlinks,%
           linkcolor=blue,%
               hyperindex,%
         plainpages=false,%
        pdfstartview=FitH]{hyperref}
\fi\fi
%\usepackage{hyperref}

% other packages
%--------------------------------------------------------------------------
\usepackage{graphicx, color}
\usepackage{subfig}
\usepackage{tikz}
\usetikzlibrary{matrix,positioning}

\usepackage{amsmath, amsthm, amssymb} % for math
\usepackage{exercise} % for exercise
\usepackage{import} % for nested input

%
% for programming
%
\usepackage{verbatim}
\usepackage{listings}
\usepackage{lipsum}
%\usepackage{algorithmic} %old version; we can use algorithmicx instead
\usepackage{algorithm}
\usepackage[noend]{algpseudocode} %for pseudo code, include algorithmicsx automatically
\usepackage{appendix}
\usepackage{makeidx} % for index support
\usepackage{titlesec}

\usepackage{fontspec}
\usepackage{xunicode}
\usepackage{fontenc}
\usepackage{textcomp}
\usepackage{url}
\usepackage{courier}

\titleformat{\paragraph}
{\normalfont\normalsize\bfseries}{\theparagraph}{1em}{}
\titlespacing*{\paragraph}
{0pt}{3.25ex plus 1ex minus .2ex}{1.5ex plus .2ex}

\lstdefinelanguage{Smalltalk}{
  morekeywords={self,super,true,false,nil,thisContext}, % This is overkill
  morestring=[d]',
  morecomment=[s]{"}{"},
  alsoletter={\#:},
  escapechar={!},
  literate=
    {BANG}{!}1
    {UNDERSCORE}{\_}1
    {\\st}{Smalltalk}9 % convenience -- in case \st occurs in code
    % {'}{{\textquotesingle}}1 % replaced by upquote=true in \lstset
    {_}{{$\leftarrow$}}1
    {>>>}{{\sep}}1
    {^}{{$\uparrow$}}1
    {~}{{$\sim$}}1
    {-}{{\sf -\hspace{-0.13em}-}}1  % the goal is to make - the same width as +
    %{+}{\raisebox{0.08ex}{+}}1		% and to raise + off the baseline to match -
    {-->}{{\quad$\longrightarrow$\quad}}3
	, % Don't forget the comma at the end!
  tabsize=2
}[keywords,comments,strings]

%% \lstdefinestyle{Haskell}{
%%   flexiblecolumns=false,
%%   basewidth={0.5em,0.45em},
%%   morecomment=[l]--,
%%   literate={+}{{$+$}}1 {/}{{$/$}}1 {*}{{$*$}}1 {=}{{$=$}}1
%%            {>}{{$>$}}1 {<}{{$<$}}1 {\\}{{$\lambda$}}1
%%            {\\\\}{{\char`\\\char`\\}}1
%%            {->}{{$\rightarrow$}}2 {>=}{{$\geq$}}2 {<-}{{$\leftarrow$}}2
%%            {<=}{{$\leq$}}2 {=>}{{$\Rightarrow$}}2
%%            {\ .}{{$\circ$}}2 {\ .\ }{{$\circ$}}2
%%            {>>}{{>>}}2 {>>=}{{>>=}}2
%%            {|}{{$\mid$}}1
%% }

% For better Haskell code outlook
\lstdefinelanguage{Haskell}{
  flexiblecolumns=false,
  basewidth={0.5em,0.45em},
  morecomment=[l]--,
  morekeywords={case, class, do, else, True, False, if, import,
    instance, module, type, data, deriving, where},
  literate={+}{{$+$}}1 {/}{{$/$}}1 {*}{{$*$}}1 {=}{{$=$}}1
           {>}{{$>$}}1 {<}{{$<$}}1 {\\}{{$\lambda$}}1
           {\\\\}{{\char`\\\char`\\}}1
           {->}{{$\rightarrow$}}2 {>=}{{$\geq$}}2 {<-}{{$\leftarrow$}}2
           {<=}{{$\leq$}}2 {=>}{{$\Rightarrow$}}2
           {\ .}{{$\circ$}}2 {\ .\ }{{$\circ$}}2
           {>>}{{>>}}2 {>>=}{{>>=}}2
           {|}{{$\mid$}}1
}[keywords,comments,strings]

% "define" Scala
\lstdefinelanguage{Scala}{
  morekeywords={abstract,case,catch,class,def,%
    do,else,extends,false,final,finally,%
    for,if,implicit,import,match,mixin,%
    new,null,object,override,package,%
    private,protected,requires,return,sealed,%
    super,this,throw,trait,true,try,%
    type,val,var,while,with,yield},
  otherkeywords={=>,<-,<\%,<:,>:,\#,@},
  sensitive=true,
  morecomment=[l]{//},
  morecomment=[n]{/*}{*/},
  morestring=[b]",
  morestring=[b]',
  morestring=[b]"""
}

\lstloadlanguages{Java, C, C++, Lisp, Haskell, Python, Smalltalk, Scala}

\lstset{
  basicstyle=\small,
  commentstyle=\rmfamily,
  %keywordstyle=\bfseries,
  texcl=true,
  showstringspaces = false,
  upquote=true,
  flexiblecolumns=false
}

\renewcommand{\lstlistingname}{Code}

% ======================================================================

\def\BibTeX{{\rm B\kern-.05em{\sc i\kern-.025em b}\kern-.08em
    T\kern-.1667em\lower.7ex\hbox{E}\kern-.125emX}}

%
% mathematics
%
\newcommand{\be}{\begin{equation}}
\newcommand{\ee}{\end{equation}}
\newcommand{\bmat}[1]{\left( \begin{array}{#1} }
\newcommand{\emat}{\end{array} \right) }
\newcommand{\VEC}[1]{\mbox{\boldmath $#1$}}

% numbered equation array
\newcommand{\bea}{\begin{eqnarray}}
\newcommand{\eea}{\end{eqnarray}}

% equation array not numbered
\newcommand{\bean}{\begin{eqnarray*}}
\newcommand{\eean}{\end{eqnarray*}}

\newtheorem{theorem}{Theorem}[section]
\newtheorem{lemma}[theorem]{Lemma}
\newtheorem{proposition}[theorem]{Proposition}
\newtheorem{corollary}[theorem]{Corollary}

\setcounter{tocdepth}{4}
\setcounter{secnumdepth}{4}


\setcounter{page}{1}

\begin{document}

%--------------------------

% ================================================================
%                 COVER PAGE
% ================================================================

\title{The evolution of insertion sort}

\author{Larry~LIU~Xinyu
\thanks{{\bfseries Larry LIU Xinyu } \newline
  Email: liuxinyu95@gmail.com \newline}
  }

\maketitle
\fi

\markboth{The evolution of insertion sort}{Elementary Algorithms}

\ifx\wholebook\relax
\chapter{The evolution of insertion sort}
\numberwithin{Exercise}{chapter}
\fi

% ================================================================
%                 Introduction
% ================================================================
\section{Introduction}
\label{introduction} \index{insertion sort}
In previous chapter, we introduced the 'hello world' data structure,
binary search tree. In this chapter, we explain insertion sort,
which can be think of the 'hello world' sorting algorithm
\footnote{Some reader may argue that 'Bubble sort' is the easiest
sort algorithm. Bubble sort isn't covered in this book as we don't
think it's a valuable algorithm\cite{wiki-bubble-sort}}.
It's straightforward, but the performance is not as good as some
divide and conqueror sorting approaches, such as quick sort
and merge sort. Thus insertion sort is seldom used as generic
sorting utility in modern software libraries. We'll analyze the
problems why it is slow, and trying to improve it bit by bit till
we reach the best bound of comparison based sorting algorithms, $O(n \lg n)$,
by evolution to tree sort. And we finally show the connection between
the 'hello world' data structure and 'hello world' sorting algorithm.

The idea of insertion sort can be vivid illustrated by a real life
poker game(pp15 - 19 in \cite{CLRS}). Suppose the cards are shuffled, and a player starts
taking card one by one.

At any time, all cards in player's hand are well sorted. When the player
gets a new card, he insert it in proper position according to the order
of points. Figure \ref{fig:hand-of-cards} shows this insertion example.

\begin{figure}[htbp]
  \centering
  \includegraphics[scale=0.5]{img/hand-of-cards.eps}
  \caption{Insert card 8 to proper position in a deck.}
  \label{fig:hand-of-cards}
\end{figure}

Based on this idea, the algorithm of insertion sort can be directly
given as the following.

\begin{algorithmic}
\Function{Sort}{$A$}
  \State $X \gets \phi$
  \For{each $x \in A$}
    \State \Call{Insert}{$X, x$}
  \EndFor
  \State \Return $X$
\EndFunction
\end{algorithmic}

It's easy to express this process with folding, which we
mentioned in the chapter of binary search tree.

\be
  sort = foldL \quad insert \quad \phi
\ee

Note that in the above algorithm, we store the sorted result in $X$,
so this isn't in-place sorting. It's easy to change it to in-place
algorithm. Denote the sequence as $A = \{a_1, a_2, ..., a_n\}$.

\begin{algorithmic}
\Function{Sort}{$A$}
  \For{$i \gets 2$ to $|A|$}
    \State insert $a_i$ to sorted sequence $\{a'_1, a'_2, ..., a'_{i-1} \}$
  \EndFor
\EndFunction
\end{algorithmic}

At any time, when we process the $i$-th element, all elements before $i$
have already been sorted. we continuously insert the current element
it consume all the unsorted data. This idea is illustrated as in figure
\ref{fig:in-place-sort}.

\begin{figure}[htbp]
  \centering
  \includegraphics[scale=0.8]{img/in-place-sort.ps}
  \caption{The left part is sorted data, continuously insert element to sorted part.}
  \label{fig:in-place-sort}
\end{figure}

We can find there is recursive concept in this definition. Thus it can
be expressed as the following.

\be
sort(A) = \left \{
  \begin{array}
  {r@{\quad:\quad}l}
  \phi & A = \phi \\
  insert(sort(\{a_2, a_3, ...\}), a_1) & otherwise
  \end{array}
\right.
\ee

% ================================================================
% Insertion
% ================================================================
\section{Insertion}
\index{insertion sort!insertion}

We haven't answered the question about how to realize insertion however.
It's a mystery how does human locate the proper position so quickly.

For computer, it's an obvious option to perform a scan. We can either
scan from left to right or vice versa. However, if the sequence is
stored in plain array, it's necessary to scan from right to left.

\begin{algorithmic}
\Function{Sort}{$A$}
  \For{$i \gets 2$ to $|A|$}
    \Comment{Insert $A[i]$ to sorted sequence $A[1...i-1]$}
    \State $x \gets A[i]$
    \State $j \gets i-1$
    \While{$j > 0 \land x < A[j]$ }
      \State $A[j+1] \gets A[j]$
      \State $j \gets j - 1$
    \EndWhile
    \State $A[j+1] \gets x$
  \EndFor
\EndFunction
\end{algorithmic}

One may think scan from left to right is natural. However, it isn't
as effect as above algorithm for plain array. The reason is that, it's
expensive to insert an element in arbitrary position in an array.
As array stores elements continuously, if we want to insert new element
$x$ at position $i$, we must shift all elements after $i$, including
$i+1, i+2, ...$ one position to right. After that the cell at position $i$
is empty, and we can put $x$ in it. This is illustrated in
figure \ref{fig:array-shift}.

\begin{figure}[htbp]
  \centering
  \includegraphics[scale=0.7]{img/array-shift.ps}
  \caption{Insert $x$ to array $A$ at position $i$.}
  \label{fig:array-shift}
\end{figure}

If the length of array is $n$, this indicates we need examine the
first $i$ elements, then perform
$n-i+1$ moves, and then insert $x$ to the $i$-th cell. So insertion
from left to right need traverse the whole array anyway.
While if we scan from right to
left, we examine $i$ elements at most, and perform the same
amount of moves.

Below Java example program is a direct translation of the above algorithm. For illustration purpose, we use primative \texttt{int} as the type of the element. Refer to the exercise of this chapter about how to abstract the element type.

\lstset{language=Java}
\begin{lstlisting}
int[] sort(int[] xs) {
    for(int i = 1; i < xs.length; ++i) {
        int x = xs[i], j = i - 1;
        while(j >=0 && x < xs[j])
            xs[j+1] = xs[j--];
        xs[j+1] = x;
    }
    return xs;
}
\end{lstlisting}

There are some other equivalent programs, for instance the following
Java example, however, this version isn't as effective as the previous one.

\lstset{language=Java}
\begin{lstlisting}
void isort(int[] xs) {
    int i, j;
    for (i = 1; i < xs.length; ++i)
        for (j = i - 1; j >= 0 && xs[j + 1] < xs[j]; --j)
            swap(xs, j, j + 1);
}
\end{lstlisting}

This is because the swapping function, which can exchange two elements
typically uses a temporary variable like the following:

\begin{lstlisting}
void swap(int[] xs, int i, int j) {
    int tmp = xs[i];
    xs[i] = xs[j];
    xs[j] = tmp;
}
\end{lstlisting}

This program presented above takes $3m$ times assignment, where $m$
is the number of inner loops. While the previous example
program uses shift operation instead of swapping. There are $m+2$ times
assignment.

We can also provide \textproc{Insert}() function explicitly, and call it
from the general insertion sort algorithm in previous section. We skip
the detailed realization here and left it as an exercise.

All the insertion algorithms are bound to $O(n)$, where $n$ is the length of
the sequence. No matter what difference among them, such as scan from left
or from right. Thus the over all performance for insertion sort is quadratic
as $O(n^2)$.

\begin{Exercise}

\begin{itemize}
\item Provide explicit insertion function, and call it with general
insertion sort algorithm. Please realize it in both procedural way and
functional way.
\item Using Java generic to abstract the element type for insertion sort. The
type should be comparable.
\end{itemize}

\end{Exercise}

% ================================================================
% Improvement 1
% ================================================================

\section{Improvement 1}
\index{Insertion sort!binary search}

Let's go back to the question, why human being can find the proper
position for insertion so quickly. We have shown a solution based on scan.
Note the fact that at any time, all cards at hands have been well sorted,
another possible solution is to use binary search to find that location.

We'll explain the search algorithms in other dedicated chapter. Binary
search is just briefly introduced for illustration purpose here.

The algorithm will be changed to call a binary search procedure.

\begin{algorithmic}
\Function{Sort}{$A$}
  \For{$i \gets 2$ to $|A|$}
    \State $x \gets A[i]$
    \State $p \gets $ \Call{Binary-Search}{$A[1...i-1], x$}
    \For{$j \gets i$ down to $p$}
      \State $A[j] \gets A[j-1]$
    \EndFor
    \State $A[p] \gets x$
  \EndFor
\EndFunction
\end{algorithmic}

Instead of scan elements one by one, binary search utilize the information
that all elements in slice of array $\{A_1, ..., A_{i-1} \}$ are sorted.
Let's assume
the order is monotonic increase order. To find a position $j$ that satisfies
$A_{j-1} \leq x \leq A_{j}$. We can first examine the middle element, i.e. $A_{\lfloor i/2 \rfloor}$. If $x$ is less than it, we need next recursively
perform binary search in the first half of the sequence; otherwise, we
only need search in the second half.

Every time, we halve the elements to be examined. This search process runs
$O(\lg n)$ time to locate the insertion position.

\begin{algorithmic}
\Function{Binary-Search}{$A, x$}
  \State $l \gets 1$
  \State $u \gets 1+|A|$
  \While{$l < u$}
    \State $m \gets \lfloor \frac{l+u}{2} \rfloor$
    \If{$A[m] = x$}
      \State \Return $m$ \Comment{Find a duplicated element}
    \ElsIf{$A[m] < x$}
      \State $l \gets m+1$
    \Else
      \State $u \gets m$
    \EndIf
  \EndWhile
  \State \Return $l$
\EndFunction
\end{algorithmic}

The improved insertion sort algorithm is still bound to $O(n^2)$,
compare to previous section, which we use $O(n^2)$ times comparison and
$O(n^2)$ moves, with binary search, we just use $O(n \lg n)$ times
comparison and $O(n^2)$ moves.

The Java example program regarding to this algorithm is given below.

\lstset{language=Java}
\begin{lstlisting}
void insertSort(int[] xs) {
    for(int i = 1; i < xs.length; ++i) {
        int x = xs[i];
        int p = binarySearch(xs, i, x);
        for (int j = i; j > p; --j)
            xs[j] = xs[j - 1];
        xs[p] = x;
    }
}

int binarySearch(int[] xs, int u, int x) {
    int l = 0;
    while (l < u) {
        int m = l + (u - l) / 2;    // == (l + u) / 2
        if (xs[m] == x) return m;
        else if (xs[m] < x) l = m + 1;
        else u = m;
    }
    return l;
}
\end{lstlisting}

\begin{Exercise}
Write the binary search in recursive manner. You needn't use purely functional
programming language.
\end{Exercise}

% ================================================================
% Improvement 2
% ================================================================

\section{Improvement 2}
\index{Insertion sort!linked-list setting}

Although we improved the search time to $O(n \lg n)$ in previous section, the
number of moves is still $O(n^2)$. The reason why it takes so many movements
is because the sequence is stored in plain array. The nature of array
is continuously layout data structure, so the insertion operation is expensive.
This hints us that we can use linked-list setting to represent the sequence.
It can improve the insertion operation from $O(n)$ to constant time $O(1)$.

\be
  insert(A, x) = \left \{
  \begin{array}
  {r@{\quad:\quad}l}
  \{ x \} & A = \phi \\
  \{ x \} \cup A & x < a_1 \\
  \{ a_1 \} \cup insert(\{ a_2, a_3, ... a_n\}, x)& otherwise
  \end{array}
\right.
\ee

Below example Scala program implement this insertion algorithm.

\lstset{language=Scala}
\begin{lstlisting}
def insert[A <% Ordered[A]] (xs : List[A], x : A) : List[A] =
  xs match {
    case List() => List(x)
    case y :: ys => if (x < y ) x :: xs else y :: insert(ys, x)
  }
\end{lstlisting}

And we can complete the two versions of insertion sort program based on
the first two equations in this chapter.

\begin{lstlisting}
def isort[A <% Ordered[A]] (xs : List[A]) : List[A] =
  xs match {
    case List() => List()
    case y :: ys => insert(isort(ys), y)
  }
\end{lstlisting}

Or we can represent the recursion with folding.

\begin{lstlisting}
  def insertSort[A <% Ordered[A]] (xs : List[A]) : List[A] =
    ((List(): List[A]) /: xs) (insert)
\end{lstlisting}

Linked-list setting solution can also be described imperatively. Suppose
function \textproc{Key}($x$), returns the value of element stored in node
$x$, and \textproc{Next}($x$) accesses the next node in the linked-list.

\begin{algorithmic}
\Function{Insert}{$L, x$}
  \State $p \gets$ NIL
  \State $H \gets L$
  \While{$L \neq$ NIL $\land $ \Call{Key}{$L$} $<$ \Call{Key}{$x$}}
    \State $p \gets L$
    \State $L \gets $ \Call{Next}{$L$}
  \EndWhile
  \State \Call{Next}{$x$} $\gets L$
  \If{$p \neq$ NIL}
    \State $H \gets x$
  \Else
    \State \Call{Next}{$p$} $\gets x$
  \EndIf
  \State \Return $H$
\EndFunction
\end{algorithmic}

For example in Java, the linked-list can be defined as the following.

\lstset{language=Java}
\begin{lstlisting}
public class Node {
    public int key;
    public Node next;
}
\end{lstlisting}

Thus the insert function can be given as below.

\begin{lstlisting}
Node insert(Node list, Node node) {
    Node prev = null;
    Node head = list;
    while (list != null && list.key < node.key) {
        prev = list;
        list = list.next;
    }
    node.next = list;
    if (prev == null)
        head = node;
    else
        prev.next = node;
    return head;
}
\end{lstlisting}

Instead of using explicit linked-list such as by pointer or reference
based structure. Linked-list can also be realized by another index array.
For any element $A[i]$, $Next[i]$ stores the index of next element
follows $A[i]$. It means $A[Next[i]]$ is the next element after $A[i]$.
We also need a variable to record the head of linked-list. It is the
index $i_0$, that $A[i_0]$ is the first element in the list. We cas use
$Next[\perp] = i_0$ for this porpuse.

\begin{figure}[htbp]
  \centering
  \begin{tikzpicture}[scale=0.8]
    \draw (-1, 0.5) node (next) {$Next$}
          (0, 0) rectangle (1, 1) node (n1) [pos=.5] {5}
          (1, 0) rectangle (2, 1) node (n2) [pos=.5] {3}
          (2, 0) rectangle (3, 1) node (n3) [pos=.5] {1}
          (3, 0) rectangle (4, 1) node (n4) [pos=.5] {-1}
          (4, 0) rectangle (5, 1) node (n5) [pos=.5] {4}
          (5, 0) rectangle (6, 1) node (n6) [pos=.5] {2}
          (5.5, -1) node (hd) {$\perp$};
    \draw (-1, 2.5) node (A) {$A$}
          (0, 2) rectangle (1, 3) node (a1) [pos=.5] {3}
          (1, 2) rectangle (2, 3) node (a2) [pos=.5] {1}
          (2, 2) rectangle (3, 3) node (a3) [pos=.5] {2}
          (3, 2) rectangle (4, 3) node (a4) [pos=.5] {5}
          (4, 2) rectangle (5, 3) node (a5) [pos=.5] {4};
    \draw[thick, ->] (n6) edge [bend left] (a2)
                 %(a2) edge (n2)
                 (n2) edge [bend right] (a3)
                 %(a3) edge (n3)
                 (n3) edge [bend left] (a1)
                 %(a1) edge (n1)
                 (n1) edge [bend right] (a5)
                 %(a5) edge (n5)
                 (n5) edge [bend left] (a4)
                 %(a4) edge (n4)
                 (hd) edge (n6);
  \end{tikzpicture}
  \caption{} \label{fig:indexed-link-list-isort}
\end{figure}

The insertion algorithm based on this solution is given like below.

\begin{algorithmic}
\Function{Insert}{$A, Next, i$}
  \State $j \gets \perp$
  \While{$Next[j] \neq$ NIL $\land A[Next[j]] < A[i]$}
    \State $j \gets Next[j]$
  \EndWhile
  \State $Next[i] \gets Next[j]$
  \State $Next[j] \gets i$
\EndFunction
\end{algorithmic}

Here $\perp$ is the special head index, that $Next[\perp]$ gives the
which is the first element in $A$. When implement this algorithm, we
can allocate the $Next$ array one extra element longer than $A$, and
use the last one to index the head element.

Below Java example program implements this algorithm.

\lstset{language=Java}
\begin{lstlisting}
public int[] insSort(int[] xs) {
    int[] next = new int[xs.length + 1]; //the last cell point to the head
    Arrays.fill(next, -1);
    for (int i = 0; i < xs.length; ++i)
        ins(xs, next, i);
    return next;
}

void ins(int[] xs, int[] next, int i) {
    int j = xs.length;  //traverse from head
    while (next[j] != -1 && xs[next[j]] < xs[i])
        j = next[j];
    next[i] = next[j];
    next[j] = i;
}
\end{lstlisting}

Although we change the insertion operation to constant time by using
linked-list. However, we have to traverse the list to find the insertion
position, which results $O(n^2)$ times comparison. This is because
linked-list, unlike array, doesn't support random access. It means we
can't use binary search with linked-list setting.

\begin{Exercise}
\begin{itemize}
\item Complete the insertion sort by using linked-list insertion function.
\item The index based linked-list return the sequence of rearranged index
as result. Write a program to re-order the original array of elements from
this result.
\end{itemize}
\end{Exercise}

% ================================================================
% Final improvement
% ================================================================

\section{Final improvement by binary search tree}
\index{Insertion sort!binary search tree}

It seems that we drive into a corner. We must improve both the comparison
and the insertion at the same time, or we will end up with $O(n^2)$ performance.

We must use binary search, this is the only way to improve the comparison
time to $O(\lg n)$. On the other hand, we must change the data structure,
because we can't achieve constant time insertion at a position with
plain array.

This remind us about our 'hello world' data structure, binary search tree.
It naturally support binary search from its definition. At the same time,
We can insert a new node in binary search tree in $O(1)$ constant time
if we already find the location.

So the algorithm changes to this.

\begin{algorithmic}
\Function{Sort}{$A$}
  \State $T \gets \phi$
  \For{each $x \in A$}
    \State $T \gets $ \Call{Insert-Tree}{$T, x$}
  \EndFor
  \State \Return \Call{To-List}{$T$}
\EndFunction
\end{algorithmic}

Where \textproc{Insert-Tree}($T, x$) and \textproc{To-List}($T$) are described in
previous chapter about binary search tree.

As we have analyzed for binary search tree, the performance of tree sort
is bound to $O(n \lg n)$, which is the lower limit of comparison based
sort(pp180 - 193 in \cite{Knuth-V3}, pp167 in \cite{CLRS}).

\section{Short summary}
In this chapter, we present the evolution process of insertion sort. Insertion
sort is well explained in most textbooks as the first sorting algorithm.
It has simple and straightforward idea, but the performance is quadratic.
Some textbooks stop here, but we want to show that there exist ways to improve
it by different point of view. We first try to save the comparison time
by using binary search, and then try to save the insertion operation by
changing the data structure to linked-list. Finally, we combine these
two ideas and evolve insertion sort to tree sort.

\ifx\wholebook\relax \else
\begin{thebibliography}{99}

\bibitem{wiki-bubble-sort}
http://en.wikipedia.org/wiki/Bubble\_sort

\bibitem{CLRS}
Thomas H. Cormen, Charles E. Leiserson, Ronald L. Rivest and Clifford Stein.
``Introduction to Algorithms, Second Edition''. ISBN:0262032937. The MIT Press. 2001

\bibitem{Knuth-V3}
Donald E. Knuth. ``The Art of Computer Programming, Volume 3: Sorting and Searching (2nd Edition)''. Addison-Wesley Professional; 2 edition (May 4, 1998) ISBN-10: 0201896850 ISBN-13: 978-0201896855

\end{thebibliography}

\end{document}
\fi
